\documentclass[a4paper,12pt]{article} %article %12
\pagestyle{plain}
\usepackage[utf8]{inputenc} % Включаем поддержку UTF8
\usepackage[english,russian]{babel}  % Включаем пакет для поддержки русского языка
\usepackage{cmap}
\usepackage[T2A]{fontenc}
\usepackage{graphicx}
\usepackage{amsmath}
\usepackage{alltt}
\usepackage{tabularx}
\usepackage{indentfirst}
\usepackage{float}
\usepackage{breqn}
\usepackage{esint}
\usepackage{enumitem}
\usepackage{amssymb}
\usepackage{fancybox}
\usepackage[unicode, linkcolor=black, pdfborder={ 1 0 0 [1]}, urlcolor=blue]{hyperref}


\textwidth=17cm
\oddsidemargin=0pt
\topmargin=-2cm
\topskip=0pt
\textheight=27cm
\renewcommand
\baselinestretch{1.25} % полуторный интервал

\graphicspath{{./images/}}

\begin{document}
\renewcommand\refname{Список использованных источников}

\thispagestyle{empty}
\hskip-1.8cm
\doublebox{
\phantom{.}
\parbox{1.05\textwidth}{%
\begin{center}
{\bf МИНИСТЕРСТВО НАУКИ И ВЫСШЕГО ОБРАЗОВАНИЯ \\РОССИЙСКОЙ ФЕДЕРАЦИИ\\
ФЕДЕРАЛЬНОЕ ГОСУДАРСТВЕННОЕ АВТОНОМНОЕ\\ОБРАЗОВАТЕЛЬНОЕ  УЧРЕЖДЕНИЕ ВЫСШЕГО ОБРАЗОВАНИЯ\\[3mm]
<<НАЦИОНАЛЬНЫЙ ИССЛЕДОВАТЕЛЬСКИЙ ЯДЕРНЫЙ УНИВЕРСИТЕТ\\<<МИФИ>>\\
(Саровский физико-технический институт -- филиал НИЯУ МИФИ)\\[3mm]
ФИЗИКО-ТЕХНИЧЕСКИЙ ФАКУЛЬТЕТ\\[8mm]
{\Large ПОЯСНИТЕЛЬНАЯ ЗАПИСКА\\К МАГИСТЕРСКОЙ ДИССЕРТАЦИИ}}\\[4mm]
На тему:\\[4mm]
{\large
\begin{tabular}{c}
\parbox{\textwidth}{\phantom{.}\hskip14cm\phantom{.}}\\[-7mm]
Автоматизация подготовки материалов\\\hline
\\[-5mm]
по образовательным программам\\\hline
\end{tabular}
}
\end{center}



\vfill\vskip2cm

\noindent
\begin{tabular}{p{0.4\textwidth}p{0.3\textwidth}p{0.3\textwidth}}
СТУДЕНТ&\rule{5cm}{0.5pt}&Полетаева О. А.\\[5mm]
РУКОВОДИТЕЛЬ РАБОТЫ&\rule{5cm}{0.5pt}&Вронский М. А.\\[5mm]
РЕЦЕНЗЕНТ&\rule{5cm}{0.5pt}&Холушкин В. С.\\[5mm]
ЗАМ. РУКОВОДИТЕЛЯ&\rule{5cm}{0.5pt}&Соловьев Т. Г.\\[5mm]
\end{tabular}

\vskip5cm
\phantom.
\begin{center}
г.\ Саров,
2024 г.
\end{center}
\phantom{.}
}
}

\textheight=25cm
\newpage
\hskip-1.8cm
\doublebox{
\phantom{.}
\parbox{1.05\textwidth}{
\begin{center}
{\bf
МИНИСТЕРСТВО НАУКИ И ВЫСШЕГО ОБРАЗОВАНИЯ \\РОССИЙСКОЙ ФЕДЕРАЦИИ\\
ФЕДЕРАЛЬНОЕ ГОСУДАРСТВЕННОЕ АВТОНОМНОЕ\\ОБРАЗОВАТЕЛЬНОЕ УЧРЕЖДЕНИЕ\\ВЫСШЕГО ОБРАЗОВАНИЯ\\
<<НАЦИОНАЛЬНЫЙ ИССЛЕДОВАТЕЛЬСКИЙ ЯДЕРНЫЙ УНИВЕРСИТЕТ\\ <<МИФИ>>\\
(Саровский физико-технический институт -- филиал НИЯУ МИФИ)\\
\rule{14cm}{0.5pt}\\
ФИЗИКО-ТЕХНИЧЕСКИЙ ФАКУЛЬТЕТ}\\[8mm]
\end{center}

\begin{tabular}{p{10.5cm}{l}}
&\textbf{УТВЕРЖДАЮ}\\
&\textbf{Зам. руководителя по УР}\\[2mm]
&\rule{2.5cm}{0.5pt}{\bf Т.\ Г.\ Соловьев}\\
&\underline{25 марта} 2024 г.\\
\end{tabular}
\vskip2cm
\begin{center}
\bf
ЗАДАНИЕ\\ НА МАГИСТЕРСКУЮ ДИССЕРТАЦИЮ
\end{center}
\vskip1cm

\noindent
\begin{tabular}{p{\textwidth}}
\begin{tabular}{ll}
{Студент}&\hbox to 14.2cm{\phantom{MM} Полетаева Ольга Алексеевна\hfil}\\\cline{2-2}&\\[-8mm]
&\hskip2.4cm{\tiny фамилия, имя, отчество}
\end{tabular}\\[6mm]
\begin{tabular}{ll}
{Направление подготовки}&\hbox to 11cm{01.04.02 <<Прикладная математика и информатика>>, \hfil}\\\cline{2-2}
\multicolumn{2}{l}{магистерская программа <<Математические и информационные технологии>>}\\\hline
\end{tabular}\\[8mm]
\begin{tabular}{ll}
{Руководитель}&\hbox to 13.1cm{начальник научно-исследовательской\hfil}\\\cline{2-2}
\multicolumn{2}{l}{лаборатории ИТМФ, к.ф.-м.н. Вронский Михаил Александрович}\\\hline&\\[-8mm]
&{\tiny должность, уч. степень и звание, фамилия, имя, отчество}\\
\end{tabular}
\end{tabular}

\vskip4.2cm

\begin{center}
\bf г.\ Саров,
2024 г.
\end{center}
\phantom{.}
}}
\newpage

\noindent
\begin{tabular}{p{\textwidth}}
{\bf Тема магистерской диссертации}\underline{\parbox{10.4cm}{\ \ Автоматизация подготовки материалов}}\\
по образовательным программам\\\hline
\\[5mm]
{\bf Место выполнения}\underline{\parbox{13.1cm}{\phantom{.}\hfill\phantom{.}}}\\[3mm]
{\bf Исходные данные к диссертации}\underline{\parbox{10.2cm}{\ \ литература, указанная в работе}}\\
\\\hline\\[-4mm]
{\bf Содержание диссертации}\underline{\parbox{11.8cm}{\ \ Рассмотрение инструментов для решения задач}}\\
автоматизации подготовки материалов; разработка форматов для необходимых данных;\\\hline
наполнение данными для решения типовых задач; примеры задач автоматизации
\\\hline
подготовки материалов.
\\\hline
\end{tabular}

\vskip10mm
\begin{center}
\underline{\textbf{Допуск к защите}}
\end{center}

\noindent
К защите представляется:\\
пояснительная записка \hskip1cm\rule{2cm}{0.5pt} страниц\\
иллюстрационный материал \rule{2cm}{0.5pt} слайдов

\vskip0.7cm

\noindent
\hskip-2mm
\begin{tabular}{lcb{5cm}}
\parbox[b]{8cm}{Руководитель диссертации\\
начальник научно-исследовательской\\
лаборатории ИТМФ, к.ф.-м.н.} &\rule{3.5cm}{0.5pt}&
Вронский М. А.\\[-3mm]
{\tiny должность, уч. степень и звание}&{\tiny подпись}&{\tiny фамилия, имя, отчество}\\[4mm]
\parbox[b]{8cm}{Рецензент диссертации\\
заведующий кафедрой\\
к.ф.-м.н.} &\rule{3.5cm}{0.5pt}&
Холушкин В. С.\\[-3mm]
{\tiny должность, уч. степень и звание}&{\tiny подпись}&{\tiny фамилия, имя, отчество}\\[4mm]
\end{tabular}

\noindent
Студент\underline{\hbox to 8.3cm{\hfil гр. МИТ-22 Полетаева Ольга Алексеевна\hfil}} допущена к защите магистерской \\[-3mm]
\phantom{.}\hskip3cm{\tiny индекс группы, фамилия, имя, отчество}\\
диссертации по направлению подготовки 01.04.02 «Прикладная математика и\\информатика», магистерская программа <<Математические и информационные \\технологии>>

\vskip1cm
\noindent{Дата защиты}\underline{ \hskip2cm июня} 2024 г.

\vskip2cm
\noindent\hskip-2mm
\begin{tabular}{lcb{5cm}}
\parbox[b]{8cm}{Зам. руководителя по УР} &\rule{3.5cm}{0.5pt}&
Т. Г. Соловьев
\end{tabular}

\newpage

\tableofcontents

\newpage

\section*{Введение}
\addcontentsline{toc}{section}{Введение}


Сопровождение образовательных программ высшего образования сопряжено с ведением достаточно большого объёма документов, связанных как с оперативными вопросами деятельности учреждения высшего образования, так и регламентацией его работы в средне- или долгосрочной перспективе. К последним относятся, в частности, рабочие учебные планы, рабочие программы дисциплин, фонды оценочных средств, годовые планы работы преподавателей, и пр.

В настоящее время сложилась традиция подготовки большинства такого рода документов с помощью современных офисных пакетов. Подготовленные таким образом документы имеют ряд недостатков:

- невозможность автоматической обработки целого набора (в связи с закрытостью форматов);

- плавающее форматирование;

- высокая трудоёмкость переработки при изменении требований.

Кроме того, в последнее время остро встал вопроса отказа от наиболее продвинутых и популярных офисных продуктов фирмы Microsoft.

В настоящей работе предлагается альтернативный подход к подготовке и сопровождению массивов документов для сопровождения программ высшего профессионального образования, основанный на:

- разработке формата описания рабочих программ дисциплин в виде текстовых файлов, размеченных с использованием формата JSON;

- скриптах обработки полученных файлов для решения типовых задач на языке Python;

- формировании необходимых документов в виде \LaTeX-файлов.

Последний этап, главным образом, предназначен для включения возможностей сложного автоматического форматирования, подготовки таблиц, использования спецсимволов, формул (что необходимо, например, для описания рабочих программ по математическим дисциплинам). Вместе с тем, целый ряд задач в рамках указанного подхода может быть решён за счёт формирования документов в виде текстовых файлов с простым форматированием, либо CSV-файлов, которые легко обработать в редакторах таблиц.

{\bf Актуальность}

Управление вузом в современных условиях невозможно без комплексной автоматизации всех аспектов его деятельности. В настоящее время наблюдается недостаток доступных и тиражируемых решений автоматизации процесса подготовки материалов по образовательным программам вуза. Использование средств автоматизации в работе на сегодняшний день является актуальным, так как это позволит создавать условия для рационального использования рабочего времени сотрудников кафедры, обеспечить быстрый доступ к необходимой информации, а также ее надежное хранение и дальнейшее использование. Современное состояние среды документооборота обусловлено не только социально-экономическими преобразованиями, но и развитием информационных технологий. Большое значение в современном управлении приобретают компьютерные технологии и средства, обеспечивающие на базе действующего законодательства и других правовых норм оперативность фиксации, сбора, обработки, поиска и передачи информации, надежность ее хранения, удаленный доступ, предоставление информации в нужное время, на нужном носителе и в нужной форме.

В условиях растущих объемов учебно-методической работы и продолжающихся изменений требований преподавателям и руководителям учебных подразделений достаточно сложно обеспечить необходимое качество учебно-методической документации.

В соответствии с Федеральными государственными образовательными стандартами третьего и четвертого поколений (\cite{FGOS}, \cite{OS}) для обеспечения учебного процесса требуется большое число разнообразных видов документов. Их составление, проверка и утверждение представляют собой достаточно трудоемкий процесс, требующий существенных затрат временных ресурсов.

Целью исследовательской работы является автоматизация подготовки материалов по образовательным программам. Создание такого программного продукта является актуальной задачей, поскольку необходимо систематизировать имеющуюся информацию в единой базе данных для качественной поддержки процессов по учебной деятельности институтов/факультетов.

Для достижения данной цели необходимо решить следующие задачи:

- разработка формата описания рабочих программ дисциплин в виде текстовых файлов, размеченных с использованием формата JSON;

-  обработки полученных файлов для решения типовых задач на языке Python;

- формирование необходимых документов в виде LaTeX-файлов.

В рамках данного исследования использовались общенаучные и специфические методы научного познания процесса создания автоматизированной информационной системы, к числу которых относятся: анализ нормативно-правовой, научной, методической литературы; метод анализа документов; системный анализ.

\newpage
\section{Описание необходимых инструментов}
\subsection{Описание формата JSON}
За представление структурированных данных на основе синтаксиса JavaScript отвечает стандартный текстовый формат под названием JSON (\cite{JSON}), аббревиатура которого расшифровывается как JavaScript Object Notation.

JSON – широко используемый текстовый формат данных. Многие среды разработки или программы содержат инструменты для его чтения и записи.

При работе с рассматриваемым текстовым форматом необходимо учитывать правила создания его структуры в объекте, массиве и при присвоении значения. Объект представляется одной из строк типа

\verb/{}/

{\tt \{\textvisiblespace строка\textvisiblespace:\textvisiblespace значение\}}

{\tt \{\textvisiblespace строка1\textvisiblespace:\textvisiblespace значение1,\textvisiblespace строка2\textvisiblespace:\textvisiblespace значение2,\textvisiblespace\ldots,\textvisiblespace строкаN\textvisiblespace:\textvisiblespace значениеN\textvisiblespace\}}

Значение может быть одним из следующих:
\begin{verbatim}
строка
число
объект
массив
true
false
null
\end{verbatim}
Таким образом, значение само может быть объектом JSON.

Массив представляет собой упорядоченную совокупность данных и находится внутри квадратных скобок. При этом значения будут отделены друг от друга. Элементами массива являются упомянутые значения. Массив представляется строкой типа
\begin{verbatim}
[ значение1, значение2, …, значениеN]
\end{verbatim}
JSON используется для обмена данными, которые являются структурированными и хранятся в файле или в строке кода. Числа, строки или любые другие объекты отображаются в виде текста, поэтому пользователь обеспечивает простое и надежное хранение информации. JSON обладает рядом преимуществ, которые и сделали его популярным:

1. Не занимает много места, является компактным в написании и быстро компилируется.

2. Создание текстового содержимого понятно человеку, просто в реализации, а чтение со стороны среды разработки не вызывает никаких проблем. Чтение может осуществляться и человеком, поскольку ничего сложного в представлении данных нет.

3. Структура преобразуется для чтения на любых языках программирования.

4. Практически все языки имеют соответствующие библиотеки или другие инструменты для чтения данных JSON.

Для создания, просмотра или редактирования JSON файла можно использовать практически любой текстовый редактор. Самый простой вариант для Windows – встроенный в операционную систему Блокнот. Достаточно удобным вариантом в этой ОС является известный текстовый редактор с поддержкой синтаксиса разных языков программирования Notepad++. Кроме того, можно использовать для этого практически любую среду разработки, а также другие операционные системы. В качестве расширения JSON файлов будем, как обычно, использовать .json.

Поскольку целей хранения и автоматической обработки информации по рабочим программам дисциплин нам потребуется текст на русском языке, важно зафиксировать выбор используемой кодировки.

\subsection{Кодировка UTF-8}

Кодирование символов – это процедура присваивания номеров графическим символам, особенно письменным. С помощью такой операции можно хранить, преобразовывать, а также передавать данные посредством цифровых технологий (компьютеров). Числовые значения, формирующие кодировку символов, называются «кодовыми точками». В совокупности они образовывают «кодовое пространство» или «карту символов».
Кодовая страница – это таблица, которая сопоставляет каждому значению байта тот или иной символ (или его отсутствие). Обычно код символа имеет размер 8 бит. Это приводит к тому, что кодовая страница может включать в себя 256 символов. Некоторые из них используются как управляющие элементы.

Сегодня в компьютерной технике используются самые разные кодировки. С помощью них удается расширить количество поддерживаемых символов. Данный момент имеет огромную значимость для латиницы и других языковых алфавитов. Особое внимание нужно уделить стандарту UTF-8 (\cite{UTF-8}). Он активно применяется в современной компьютерной технике, а также в программировании.
UTF-8 – это еще один тип Unicode-стандарта. Данный вариант кодировки предусматривает в своем составе кириллицу. Называется UTF-8 кодировкой переменной длины. Несмотря на 8 в названии стандарта, длина действительно меняется. Каждый символ может получить код, длина которого составит от 1 до 6 байт. Обычно стандарт использует записи длиной до 4 байт. Латинские буквы содержатся в одном байте – точно так же, как и в случае с ASCII.

\subsection{Python}

Python (\cite{Python}) — это скриптовый язык программирования. Он является интерпретируемым, поэтому -- относительно медленным при работе. Однако, он содержит значительное количество библиотек, в частности, библиотеку \verb/json/ для работы с файлами в этом формате. Кроме того, в нём есть целый ряд полезных функций для работы со строками, связанными с их разбором, заменой подстрок, разбиением и пр.

Приведём простой пример.
Пусть в файл \verb'test.json' формате JSON записаны следующие данные
\begin{verbatim}
{
  "A" : {
     "alpha" : [ 1,2,3],
     "beta" : "alphabet"
   },
  "C" : { "g" : "gamma", "d" : "delta" }
}
\end{verbatim}
Если запустить скрипт на Python
\begin{verbatim}
import json
INP = open( "test.json","r")
s = json.load( INP )
print(s["A"]["alpha"])
print(s["A"]["beta"])
print(s["C"]["g"])
print(s["C"]["d"])
\end{verbatim}
то выдача будет
\begin{verbatim}
[1, 2, 3]
alphabet
gamma
delta
\end{verbatim}
То есть, с данными в формате JSON изнутри Python можно работать, как с ассоциативным массивом.


\subsection{Описание формата CSV}

CSV (\cite{CSV}, Comma-Separated Values) — это текстовый формат для представления табличных данных. Строка таблицы соответствует строке текста, которая содержит поля, разделенные запятыми. Тип файлов предназначен для передачи объемных текстовых данных между различными программами и сервисами.

Файлы этого типа предназначены для передачи информации, как правило, большого объема, между различными программами и сервисами. Например, средствами хранения баз данных, ридеров и редакторов текста и электронных таблиц.

Иногда к CSV относят не только сам этот формат, но и другие – схожие по спецификациям. В частности, TSV (tab-separated values) – таблицы, где поля с данными разделяются табуляцией, и SCSV (semicolon separated values), где в качестве разделителей пишут точку с запятой. В прочих близкородственных форматах для разграничения полей используют кавычки, двоеточие, вертикальную черту, определенную последовательность символов и т. д.

Такое широкие понимание стандартов CSV может приводить к изменению структуры и содержимого таблиц при переносе файлов из программы в программу, а также ограничивать возможность их просмотра, если открывать не в том приложении, в котором они были созданы.

Поэтому при создании и редактировании файлов этого типа важно соблюдать правила форматирования и брать во внимание некоторые их особенности.

\subsection{Система подготовки документов \LaTeX}

\LaTeX{} (\cite{LaTeX}) -- <<де факто>> стандарт для работы со сложными документами, в которых важна корректностью отображения любых данных. То есть \LaTeX{} надо знать студентам, аспирантам, преподавателям, ученым и любым другим специалистам, которым нужно сделать качественную верстку текста.

Работа с \LaTeX{} сводится к написанию кода, в котором есть несколько базовых компонентов и фрагментов. При этом, надо знать несколько нюансов:

Внутритекстовые формулы окружаются с обеих сторон знаками \verb'$'.

Выключные формулы (вынесенные в отдельную строку) окружаются знаками \verb'$$' или парой команд \verb'\[' и \verb'\]'. Формулы, заключенные в \verb'$$' всегда центрируются по горизонтали.

Внутритекстовые формулы, за исключением самых коротких, набираются отдельной строкой.

Для выравнивания документа служат теги: \verb'\flushleft' — по левому краю, \verb'\flushright' — по правому краю, \verb'\center' — по центру.

Окружение verbatim по умолчанию помогает корректно отображать даже сложный программный код.

Большинство математических функций и символов, а также скобки, матрицы и системы уравнений в синтаксисе LaTeX обозначаются тегами, начинающимися с символа \verb'\'.

Для работы с выражениями, которые занимают больше одной строки (то есть для аккуратного переноса текста или вычислений), можно использовать разделитель \verb'\\'.

Буквы греческого языка и лемниската, символ бесконечности и другие вводятся в виде тегов.

\section{Структуры основных файлов для описания программ подготовки}
\subsection{Файлы для описания компетенций}

Для описания компетенций формируются два файла в формате JSON. Первый — файл описания компетенций. Он состоит из двух массивов. Первый описывает универсальные компетенции, и включает для каждой из них:

идентификатор компетенции,

её содержание,

идентификаторы и содержание знаний, умений и навыков для данной компетенции

\begin{verbatim}
"UniversalCompetence":[
  {"Id", "Contents", "Knowledge" : { "Id", "Contents"},
   "Ability" : { "Id", "Contents"}, "Skill" : { "Id", "Contents"}  },
...
]
\end{verbatim}

Второй массив описывает профессиональные компетенции, разбитые по типам задач профессиональной деятельности, и включает для каждой такой задачи:

тип задачи профессиональной деятельности,

задачу профессиональной деятельности,

область профессиональной деятельности,

массив профессиональных компетенций для данной задачи, включающий для каждой компетенции её идентификатор, содержание, идентификаторы и содержание знаний, умений и навыков для данной компетенции, профессиональный стандарт и обобщённую трудовую функцию, для данной компетенции.
\begin{verbatim}
"ProfessionalCompetence" : [
  { "TypeTaskPD",
    "TaskPD",
    "ObjectField",
    "Competence" : [
       { "Id", "Contents", "Knowledge" : { "Id", "Contents" },
        "Ability" : { "Id", "Contents" },
        "Skill" : { "Id", "Contents" },
        "ProfStandard", "OTF" },
      ...
    ]
  },
  ...
]
\end{verbatim}

Приведём данные для компетентностной модели для образовательной программы <<Высокопроизводительные вычисления и технологии параллельного программирования в пакете Логос>> по специальности 01.04.02 Прикладная математика и информатика (см. \cite{KM}).
\begin{verbatim}
{
"UniversalCompetence":[
  {
    "Id":"УК-1",
    "Contents":"Способен осуществлять критический анализ проблемных
                ситуаций на основе системного подхода, вырабатывать
                стратегию действий",
    "Knowledge":{
      "Id":"З-УК-1",
      "Contents":"Знать: методы системного и критического анализа;
                  методики разработки стратегии действий для выявления
                  и решения проблемной ситуации "},
    "Ability":{
      "Id":"У-УК-1",
      "Contents":"Уметь: применять методы системного подхода и критического
                  анализа проблемных ситуаций; разрабатывать стратегию
                  действий, принимать конкретные решения для ее реализации "},
    "Skill":{
      "Id":"В-УК-1",
      "Contents":"Владеть: методологией системного и критического анализа
                  проблемных ситуаций; методиками постановки цели, определения
                  способов ее достижения, разработки стратегий действий"}
  },
  {
    "Id":"УК-2",
    "Contents":"Способен управлять проектом на всех этапах его жизненного цикла",
    "Knowledge":{
      "Id":"З-УК-2",
      "Contents":"Знать: этапы жизненного цикла проекта; этапы разработки и
                  реализации проекта; методы разработки и управления проектами"},
    "Ability":{
      "Id":"У-УК-2",
      "Contents":"Уметь: разрабатывать проект с учетом анализа альтернативных
                  вариантов его реализации, определять целевые этапы, основные
                  направления работ; объяснить цели и сформулировать задачи,
                  связанные с подготовкой и реализацией проекта; управлять
                  проектом на всех этапах его жизненного цикла; "},
    "Skill":{
      "Id":"В-УК-2",
      "Contents":"Владеть: методиками разработки и управления проектом;
                  методами оценки потребности в ресурсах и эффективности проекта"}
  },
  {
    "Id":"УК-3",
    "Contents":"Способен организовывать и руководить работой команды, вырабатывая
                командную стратегию для достижения поставленной цели",
    "Knowledge":{
      "Id":"З-УК-3",
      "Contents":"Знать: методики формирования команд; методы эффективного
                  руководства коллективами; основные теории лидерства и стили
                  руководства"},
    "Ability":{
      "Id":"У-УК-3",
      "Contents":"Уметь: разрабатывать план групповых и организационных
                  коммуникаций при подготовке и выполнении проекта; сформулировать
                  задачи членам команды для достижения поставленной цели;
                  разрабатывать командную стратегию; применять эффективные
                  стили руководства командой для достижения поставленной цели"},
    "Skill":{
      "Id":"В-УК-3",
      "Contents":"Владеть: умением анализировать, проектировать и организовывать
                  межличностные, групповые и организационные коммуникации в
                  команде для достижения поставленной цели; методами организации
                  и управления коллективом"}
  },
  {
    "Id":"УК-4",
    "Contents":"Способен применять современные коммуникативные технологии, в том
                числе на иностранном(ых) языке(ах), для академического и
                профессионального взаимодействия",
    "Knowledge":{
      "Id":"З-УК-4",
      "Contents":"Знать: правила и закономерности личной и деловой устной и
                  письменной коммуникации; современные коммуникативные технологии
                  на русском и иностранном языках; существующие профессиональные
                  сообщества для профессионального взаимодействия "},
    "Ability":{
      "Id":"У-УК-4",
      "Contents":"Уметь: применять на практике коммуникативные технологии,
                  методы и способы делового общения для академического и
                  профессионального взаимодействия "},
    "Skill":{
      "Id":"В-УК-4",
      "Contents":"Владеть: методикой межличностного делового общения на русском
                  и иностранном языках, с применением профессиональных языковых
                  форм, средств и современных коммуникативных технологий"}
  },
  {
    "Id":"УК-5",
    "Contents":"Способен анализировать и учитывать разнообразие культур в
                процессе межкультурного взаимодействия",
    "Knowledge":{
      "Id":"З-УК-5",
      "Contents":"Знать: закономерности и особенности социально-исторического
                  развития различных культур; особенности межкультурного
                  разнообразия общества; правила и технологии эффективного
                  межкультурного взаимодействия"},
    "Ability":{
      "Id":"У-УК-5",
      "Contents":"Уметь: понимать и толерантно воспринимать межкультурное
                  разнообразие общества; анализировать и учитывать разнообразие
                  культур в процессе межкультурного взаимодействия"},
    "Skill":{
      "Id":"В-УК-5",
      "Contents":"Владеть: методами и навыками эффективного межкультурного
                  взаимодействия"}
  },
  {
    "Id":"УК-6",
    "Contents":"Способен определять и реализовывать приоритеты собственной
                деятельности и способы ее совершенствования на основе самооценки",
    "Knowledge":{
      "Id":"З-УК-6",
      "Contents":"Знать: методики самооценки, самоконтроля и саморазвития с
                  использованием подходов здоровьесбережения; "},
    "Ability":{
      "Id":"У-УК-6",
      "Contents":"Уметь: решать задачи собственного личностного и
                  профессионального развития, определять и реализовывать
                  приоритеты совершенствования собственной деятельности;
                  применять методики самооценки и самоконтроля; применять
                  методики, позволяющие улучшить и сохранить здоровье в
                  процессе жизнедеятельности; "},
    "Skill":{
      "Id":"В-УК-6",
      "Contents":"Владеть: технологиями и навыками управления своей
                  познавательной деятельностью и ее совершенствования на
                  основе самооценки, самоконтроля и принципов самообразования
                  в течение всей жизни, в том числе с использованием
                  здоровьесберегающих подходов и методик"}
  },
  {
    "Id":"УКЦ-1",
    "Contents":"Способен решать исследовательские, научно-технические и
                производственные задачи в условиях неопределенности, в том
                числе выстраивать деловую коммуникацию и организовывать работу
                команды с использованием цифровых ресурсов и технологий в
                цифровой среде",
    "Knowledge":{
      "Id":"З-УКЦ-1",
      "Contents":"Знать современные цифровые технологии, используемые для
                  выстраивания деловой коммуникации и организации индивидуальной
                  и командной работы"},
    "Ability":{
      "Id":"У-УКЦ-1",
      "Contents":"Уметь подбирать наиболее релевантные цифровые решения для
                  достижения поставленных целей и задач, в том числе в условиях
                  неопределенности"},
    "Skill":{
      "Id":"В-УКЦ-1",
      "Contents":"Владеть навыками решения исследовательских, научно-технических
                  и производственных задач с использованием цифровых технологий"}
  },
  {
    "Id":"УКЦ-2",
    "Contents":"Способен к самообучению, самоактуализации и саморазвитию с
                использованием различных цифровых технологий в условиях их
                непрерывного совершенствования",
    "Knowledge":{
      "Id":"З-УКЦ-2",
      "Contents":"Знать основные цифровые платформы, технологи и интернет
                  ресурсы используемые при онлайн обучении;"},
    "Ability":{
      "Id":"У-УКЦ-2",
      "Contents":"Уметь использовать различные цифровые технологии для
                  организации обучения; "},
    "Skill":{
      "Id":"В-УКЦ-2",
      "Contents":"Владеть навыками самообучения, самооактулизации и саморазвития
                  с использованием различных цифровых технологий"}
  },
  {
    "Id":"ОПК-1",
    "Contents":"Способен решать актуальные задачи фундаментальной и прикладной
                математики",
    "Knowledge":{
      "Id":"З-ОПК-1",
      "Contents":"Знать актуальные задачи фундаментальной и прикладной математики,
                  методы математического моделирования"},
    "Ability":{
      "Id":"У-ОПК-1",
      "Contents":"Уметь использовать методы математического моделирования для
                  решения задач фундаментальной и прикладной математики"},
    "Skill":{
      "Id":"В-ОПК-1",
      "Contents":"Владеть методами математического моделирования и основами
                  их использования"}
  },
  {
    "Id":"ОПК-2",
    "Contents":"Способен совершенствовать и реализовывать новые математические
                методы решения прикладных задач",
    "Knowledge":{
      "Id":"З-ОПК-2",
      "Contents":"Знать основные понятия, математические методы решения прикладных
                  задач, принципы математического моделирования и методы
                  верификации. "},
    "Ability":{
      "Id":"У-ОПК-2",
      "Contents":"Уметь применять полученную теоретическую базу для решения
                  практических задач; "},
    "Skill":{
      "Id":"В-ОПК-2",
      "Contents":"Владеть основными математическими методами решения прикладных
                  задач"}
  },
  {
    "Id":"ОПК-3",
    "Contents":"Способен разрабатывать математические модели и проводить их
                анализ при решении задач в области профессиональной
                деятельности",
    "Knowledge":{
      "Id":"З-ОПК-3",
      "Contents":"Знать основные методы и принципы математического моделирования,
                  методы построения математических моделей типовых
                  профессиональных задач, способы нахождения решений
                  математических моделей и содержательной интерпретации
                  полученных результатов"},
    "Ability":{
      "Id":"У-ОПК-3",
      "Contents":"Уметь составлять математические модели типовых профессиональных
                  задач, находить способы их решения и профессионально
                  интерпретировать смысл полученного результата"},
    "Skill":{
      "Id":"В-ОПК-3",
      "Contents":"Владеть методами построения математических моделей типовых
                  профессиональных задач, способами нахождения решений
                  математических моделей и содержательной интерпретации
                  полученных результатов"}
  }
],
"ProfessionalCompetence":[
  {
    "TypeTaskPD":"Тип задачи профессиональной деятельности:
                  научно-исследовательский",
    "TaskPD":"разработка и использование математических, информационных
              и имитационных моделей по тематике выполняемых
              научно-исследовательских, опытно-конструкторских работ",
    "ObjectField":"математическое моделирование и высокопроизводительные
                   вычисления в задачах механики сплошной среды и физики
                   высоких плотностей энергии; разработка прикладных
                   программных комплексов; разработка высокопроизводительных
                   ЭВМ и программного обеспечения для них; компьютерное
                   сопровождение и обработка результатов физических
                   экспериментов",
    "Competence":[
      {
        "Id":"ПК-1",
        "Contents":"способен проводить научные исследования и получать новые
                    научные и прикладные результаты самостоятельно и в
                    составе научного коллектива",
        "Knowledge":{
          "Id":"З-ПК-1",
          "Contents":"Знать основные методы и принципы научных исследований,
                      математического моделирования, основные проблемы
                      профессиональной области, требующие использования
                      современных научных методов исследования"},
        "Ability":{
          "Id":"У-ПК-1",
          "Contents":"Уметь ставить и решать прикладные исследовательские
                      задачи; оценивать результаты исследований; формулировать
                      результаты проведенного исследования в виде конкретных
                      рекомендаций, проводить научные исследования и получать
                      новые научные и прикладные результаты самостоятельно и
                      в составе научного коллектива"},
        "Skill":{
          "Id":"В-ПК-1",
          "Contents":"Владеть навыками выбора и использования математических
                      средств научных исследований, методами анализа и синтеза
                      научной информации"},
        "ProfStandard":"Профессиональный стандарт <<40.011. Специалист по
                        научно-исследовательским и опытно-конструкторским
                        разработкам>>",
        "OTF":"B/01.6. Проведение патентных исследований и определение
               характеристик продукции (услуг)"
      },
      {
        "Id":"ПК-2",
        "Contents":"способен к разработке и внедрению наукоемкого программного
                    обеспечения, способствующего решению передовых задач науки
                    и техники на основе современных математических методов
                    и алгоритмов",
        "Knowledge":{
          "Id":"З-ПК-2",
          "Contents":"Знать текущее положение современных научных достижений,
                      современные математические методы и алгоритмы для
                      разработки наукоемкого программного обеспечения"},
        "Ability":{
          "Id":"У-ПК-2",
          "Contents":"Уметь применять современные математические методы и
                      алгоритмы для разработки наукоемкого программного
                      обеспечения"},
        "Skill":{
          "Id":"В-ПК-2",
          "Contents":"Владеть навыками разработки и внедрения наукоемкого
                      программного обеспечения"},
        "ProfStandard":"Профессиональный стандарт <<40.011. Специалист по
                        научно-исследовательским и опытно-конструкторским
                        разработкам>>",
        "OTF":"B/02.6. Проведение работ по обработке и анализу научно-технической
               информации и результатов исследований"
      },
      {
        "Id":"ПК-3",
        "Contents":"способен развивать инновационный потенциал новых научных
                    и научно-технологических разработок",
        "Knowledge":{
          "Id":"З-ПК-3",
          "Contents":"Знать основы планирования и организации научных
                      исследований в профессиональной области; методику
                      постановки задач по решению теоретических и прикладных
                      исследовательских проблем; методы и средства научных
                      исследований в профессиональной области, правила и
                      принципы научной этики, методы математического
                      моделирования"},
        "Ability":{
          "Id":"У-ПК-3",
          "Contents":"Уметь оценивать и развивать инновационный потенциал
                      новых научных и научно-технологических разработок,
                      осуществлять постановку задач по решению теоретических
                      и прикладных исследовательских проблем; составить план
                      научных исследований; выдвинуть гипотезы по направлению
                      исследований и соотнести их с полученными результатами;
                      организовать свою научно-исследовательскую работу;
                      определять методы и средства научных исследований для
                      решения конкретных задач в своей предметной области;
                      оценивать результаты исследований, использовать методы
                      математического моделирования "},
        "Skill":{
          "Id":"В-ПК-3",
          "Contents":"Владеть навыками постановки задач по решению теоретических
                      и прикладных исследовательских проблем; навыками выбора
                      и использования методов и средств научных исследований
                      задач в своей предметной области; навыками методами
                      работы с литературными источниками; методами анализа
                      результатов научных исследований; методами обобщения
                      результатов научных исследований для развития
                      инновационного потенциала новых научных и
                      научно-технологических разработок"},
        "ProfStandard":"Профессиональный стандарт <<40.011. Специалист по
                        научно-исследовательским и опытно-конструкторским
                        разработкам>>",
        "OTF":"D/01.7. Формирование новых направлений научных исследований и
               опытно-конструкторских разработок"
      },
      {
        "Id":"ПК-4",
        "Contents":"способен проводить экспертизы инновационных проектов в
                    сфере своей профессиональной деятельности",
        "Knowledge":{
          "Id":"З-ПК-4",
          "Contents":"Знать основные методы и принципы экспертизы инновационных
                      проектов в сфере своей профессиональной деятельности. "},
        "Ability":{
          "Id":"У-ПК-4",
          "Contents":"Уметь проводить экспертизы инновационных проектов,
                      оценивать перспективы развития проектов в сфере своей
                      профессиональной деятельности. "},
        "Skill":{
          "Id":"В-ПК-4",
          "Contents":"Владеть навыками проведения экспертизы инновационных
                      проектов в сфере своей профессиональной деятельности."},
        "ProfStandard":"Профессиональный стандарт <<40.011. Специалист по
                        научно-исследовательским и опытно-конструкторским
                        разработкам>>",
        "OTF":"B/02.6. Проведение работ по обработке и анализу научно-технической
               информации и результатов исследований"
      }
    ]
  },
  {
    "TypeTaskPD":"Тип задачи профессиональной деятельности:
                  нормативно-методический",
    "TaskPD":"Разработка и применение стандарнтов оформления отчётной и
              программной документации, пользовательских интерфейсов
              программ, баз данных",
    "ObjectField":"математическое моделирование и высокопроизводительные
                   вычисления в задачах механики сплошной среды и физики
                   высоких плотностей энергии; разработка прикладных программных
                   комплексов; разработка высокопроизводительных ЭВМ и
                   программного обеспечения для них; компьютерное сопровождение
                   и обработка результатов физических экспериментов",
    "Competence":[
      {
        "Id":"ПК-8",
        "Contents":"способен разрабатывать корпоративные стандарты и профили
                    функциональной стандартизации приложений, систем,
                    информационной инфраструктуры",
        "Knowledge":{
          "Id":"З-ПК-8",
          "Contents":"Знать основные цели и задачи, особенности содержания
                      корпоративных стандартов и профилей функциональной
                      стандартизации приложений, систем, информационной
                      инфраструктуры"},
        "Ability":{
          "Id":"У-ПК-8",
          "Contents":"Уметь разрабатывать корпоративные стандарты и профили
                      функциональной стандартизации приложений, систем,
                      информационной инфраструктуры"},
        "Skill":{
          "Id":"В-ПК-8",
          "Contents":"Владеть навыками разработки корпоративных стандартов
                      и профилей функциональной стандартизации приложений,
                      систем, информационной инфраструктуры."},
        "ProfStandard":"Профессиональный стандарт <<06.019. Технический писатель
                        (специалист по технической документации в области
                        информационных технологий)>>",
        "OTF":"D.6. Разработка технических документов, адресованных специалисту
               по информационным технологиям"
      }
    ]
  },
  {
    "TypeTaskPD":"Тип задачи профессиональной деятельности:
                  организационно-управленческий",
    "TaskPD":"планирование, организация и руководство исследованиями, связанными
              с применением прикладной математики и информатики в конкретных
              предметных областях",
    "ObjectField":"математическое моделирование и высокопроизводительные
                   вычисления в задачах механики сплошной среды и физики высоких
                   плотностей энергии; разработка прикладных программных
                   комплексов; разработка высокопроизводительных ЭВМ и
                   программного обеспечения для них; компьютерное сопровождение
                   и обработка результатов физических экспериментов",
    "Competence":[
      {
        "Id":"ПК-7",
        "Contents":"способен управлять проектами, планировать
                    научно-исследовательскую деятельность, анализировать риски,
                    управлять командой проекта в области прикладной математики
                    и информационных технологий",
        "Knowledge":{
          "Id":"З-ПК-7",
          "Contents":"Знать основные цели и задачи планирования
                      научно-исследовательской деятельности, основы анализа
                      рисков проекта в области прикладной математики и
                      информационных технологий; "},
        "Ability":{
          "Id":"У-ПК-7",
          "Contents":"Уметь управлять проектами, планировать
                      научно-исследовательскую деятельность, анализировать риски,
                      управлять командой проекта в области прикладной математики
                      и информационных технологий. "},
        "Skill":{
          "Id":"В-ПК-7",
          "Contents":"Владеть навыками управления проектами, планирования
                      научно-исследовательской деятельности и анализа рисков в
                      области прикладной математики и информационных
                      технологий."},
        "ProfStandard":"Профессиональный стандарт <<06.016. Руководитель
                        проектов в области информационных технологий>>",
        "OTF":"B.7 Управление проектами в области ИТ малого и среднего уровня
               сложности в условиях неопределенностей, порождаемых запросами на
               изменения, с применением формальных инструментов управления
               рисками и проблемами проекта"
      }
    ]
  },
  {
    "TypeTaskPD":"Тип задачи профессиональной деятельности: педагогический",
    "TaskPD":"организация и проведение образовательной деятельности (ВО) в
              области прикладной математики и информатики",
    "ObjectField":"математическое моделирование и высокопроизводительные
                   вычисления в задачах механики сплошной среды и физики
                   высоких плотностей энергии; разработка прикладных программных
                   комплексов; разработка высокопроизводительных ЭВМ и
                   программного обеспечения для них; компьютерное сопровождение
                   и обработка результатов физических экспериментов",
    "Competence":[
      {
        "Id":"ПК-10",
        "Contents":"способен осуществлять подготовку кадров в области
                    прикладной математики и информационных технологий",
        "Knowledge":{
          "Id":"З-ПК-10",
          "Contents":"Знать основные цели и задачи, особенности содержания и
                      организации педагогического процесса на основе
                      компетентостного подхода; психологические особенности
                      обучающихся; современные технологии диагностики и
                      оценивания качества образовательного процесса;
                      особенности педагогического взаимодействия в условиях
                      изменяющегося образовательного пространства"},
        "Ability":{
          "Id":"У-ПК-10",
          "Contents":"Уметь организовывать образовательно-воспитательный
                      процесс в изменяющихся социокультурных условиях;
                      применять психолого-педагогические знания в разных
                      видах образовательной деятельности. "},
        "Skill":{
          "Id":"В-ПК-10",
          "Contents":"Владеть навыками организации педагогического процесса
                      для подготовки и переподготовки кадров в области
                      прикладной математики и информационных технологий"},
        "ProfStandard":"Профессиональный стандарт <<40.011. Специалист по
                        научно-исследовательским и опытно-конструкторским
                        разработкам>>",
        "OTF":"D/02.7. Подготовка и повышение квалификации кадров высшей
               квалификации в соответствующей области знаний"
      },
      {
        "Id":"ПК-9",
        "Contents":"способен использовать современные информационные технологии
                    в образовательной деятельности",
        "Knowledge":{
          "Id":"З-ПК-9",
          "Contents":"Знать основные цели и задачи, особенности содержания и
                      организации педагогического процесса. "},
        "Ability":{
          "Id":"У-ПК-9",
          "Contents":"Уметь использовать современные информационные технологии
                      в образовательной деятельности. "},
        "Skill":{
          "Id":"В-ПК-9",
          "Contents":"Владеть навыками использования современных информационных
                      технологий в образовательной деятельности."},
        "ProfStandard":"Профессиональный стандарт <<01.001. Педагог
                        (педагогическая деятельность в сфере дошкольного,
                        начального общего, основного общего, среднего общего
                        образования) (воспитатель, учитель)>>",
        "OTF":"B/03.6. Педагогическая деятельность по реализации программ
               основного и среднего общего образования"
      }
    ]
  },
  {
    "TypeTaskPD":"Тип задачи профессиональной деятельности: проектный",
    "TaskPD":"разработка и реализация проектов, связанных с применением
              прикладной математики и информатики в конкретных предметных
              областях",
    "ObjectField":"математическое моделирование и высокопроизводительные
                   вычисления в задачах механики сплошной среды и физики
                   высоких плотностей энергии; разработка прикладных
                   программных комплексов; разработка высокопроизводительных
                   ЭВМ и программного обеспечения для них; компьютерное
                   сопровождение и обработка результатов физических
                   экспериментов",
    "Competence":[
      {
        "Id":"ПК-5",
        "Contents":"способен чётко формулировать цели и задачи научно-прикладных
                    проектов, разрабатывать концептуальные и теоретические
                    модели решаемых задач",
        "Knowledge":{
          "Id":"З-ПК-5",
          "Contents":"Знать основные цели и задачи научно-прикладных проектов,
                      разрабатывать концептуальные и теоретические модели
                      решаемых задач"},
        "Ability":{
          "Id":"У-ПК-5",
          "Contents":"Уметь чётко формулировать цели и задачи научно-прикладных
                      проектов, разрабатывать концептуальные и теоретические
                      модели решаемых задач; "},
        "Skill":{
          "Id":"В-ПК-5",
          "Contents":"Владеть навыками разработки теоретических моделей решаемых
                      задач"},
        "ProfStandard":"Профессиональный стандарт <<40.008. Специалист по
                        организации и управлению научно-исследовательскими и
                        опытно-конструкторскими работами>>",
        "OTF":"B/01.6. Организация выполнения научно-исследовательских работ по
               проблемам, предусмотренным тематическим планом сектора
               (лаборатории)"
      }
    ]
  },
  {
    "TypeTaskPD":"Тип задачи профессиональной деятельности:
                  производственно-технологический",
    "TaskPD":"Использование высокопроизводительных вычислений, компьютерных
              систем и сетей, электронных баз данных в научно-исследовательских,
              опытно-конструкторских, производственно-технологических работах",
    "ObjectField":"математическое моделирование и высокопроизводительные
                   вычисления в задачах механики сплошной среды и физики высоких
                   плотностей энергии; разработка прикладных программных
                   комплексов; разработка высокопроизводительных ЭВМ и
                   программного обеспечения для них; компьютерное сопровождение
                   и обработка результатов физических экспериментов",
    "Competence":[
      {
        "Id":"ПК-6",
        "Contents":"способен к проектированию и разработке наукоёмкого
                    программного обеспечения на основе технического задания",
        "Knowledge":{
          "Id":"З-ПК-6",
          "Contents":"Знать основные цели и задачи проектирования и разработки
                      наукоемкого программного обеспечения на основе
                      технического задания. "},
        "Ability":{
          "Id":"У-ПК-6",
          "Contents":"Уметь разрабатывать наукоемкое программное обеспечение
                      на основе технического задания. "},
        "Skill":{
          "Id":"В-ПК-6",
          "Contents":"Владеть навыками разработки и проектирования наукоемкого
                      программного обеспечения на основе технического задания."},
        "ProfStandard":"Профессиональный стандарт <<25.048.
                        Инженер-исследователь по прочности летательных аппаратов
                        в ракетно-космической технике при силовом и температурном
                        воздействиях>>",
        "OTF":"C/02.7. Организация и проведение расчетно-экспериментальных работ
               по исследованию прочности элементов ЛА при силовом и
               температурном воздействиях"
      }
    ]
  },
  {
    "TypeTaskPD":"Тип задачи профессиональной деятельности:
                  организационно-управленческий",
    "TaskPD":"Планирование, организация и руководство исследованиями,
              связанными с применением прикладной математики и информатики
              в конкретных предметных областях",
    "ObjectField":"математическое моделирование и высокопроизводительные
                   вычисления в задачах механики сплошной среды и физики высоких
                   плотностей энергии; разработка прикладных программных
                   комплексов; разработка высокопроизводительных ЭВМ и
                   программного обеспечения для них; компьютерное сопровождение
                   и обработка результатов физических экспериментов",
    "Competence":[
      {
        "Id":"ПК-7.1",
        "Contents":"Способен осуществлять проектирование, разработку,
                    моделирование и эксплуатацию компонентов пакета программ
                    Логос в решении задач имитационного моделирования",
        "Knowledge":{
          "Id":"З-ПК-7.1",
          "Contents":"знать основные подходы к разработке программ с
                      использованием параллельных и векторных вычислений;
                      основные современные архитектуры многопроцессорных
                      вычислительных систем; "},
        "Ability":{
          "Id":"У-ПК-7.1",
          "Contents":"уметь осуществлять проектирование, разработку,
                      моделирование и эксплуатацию пакета программ Логос
                      в решении задач имитационного моделирования; "},
        "Skill":{
          "Id":"В-ПК-7.1",
          "Contents":" владеть навыками работы в пакете программ Логос"},
        "ProfStandard":"Профессиональный стандарт <<06.001. Программист>>",
        "OTF":"D/03.6. Проектирование программного обеспечения"
      }
    ]
  },
  {
    "TypeTaskPD":"Тип задачи профессиональной деятельности: педагогический",
    "TaskPD":"Организация и проведение образовательной деятельности (ВО) в
              области прикладной математики и информатики",
    "ObjectField":"математическое моделирование и высокопроизводительные
                   вычисления в задачах механики сплошной среды и физики высоких
                   плотностей энергии; разработка прикладных программных
                   комплексов; разработка высокопроизводительных ЭВМ и
                   программного обеспечения для них; компьютерное сопровождение
                   и обработка результатов физических экспериментов",
    "Competence":[
      {
        "Id":"ПК-7.2",
        "Contents":"Способен осуществлять подготовку и переподготовку кадров
                    по использованию отечественного пакета программ Логос для
                    специалистов оборонно-промышленного комплекса",
        "Knowledge":{
          "Id":"З-ПК-7.2",
          "Contents":"Знать основные цели и задачи, особенности содержания и
                      организации педагогического процесса на основе
                      компетентностного подхода "},
        "Ability":{
          "Id":"У-ПК-7.2",
          "Contents":"осуществлять подготовку и переподготовку кадров по
                      использованию пакета программ Логос для
                      научно-исследовательских и производственных организаций
                      атомной и других высокотехнологичных отраслей
                      промышленности"},
        "Skill":{
          "Id":"В-ПК-7.2",
          "Contents":"Владеть навыками организации педагогического процесса
                      для подготовки и переподготовки кадров по использованию
                      пакета программ Логос"},
        "ProfStandard":"Профессиональный стандарт <<40.011. Специалист по
                        научно-исследовательским и опытно-конструкторским
                        разработкам>>",
        "OTF":"D/02.7. Подготовка и повышение квалификации кадров высшей
               квалификации в соответствующей области знаний"
      }
    ]
  }
]
}
\end{verbatim}

Данный файл целесообразно готовить и обрабатывать отдельно, поскольку в настоящее время компетентностный подход находится в стадии достаточно регулярной переработки, которая слабо затрагивает сутевую часть образовательных программ. В случае изменений они легко могут быть сделаны локально.

\subsection{Файл для текущего состава преподавателей и соответствия по компетенциям для всех дисциплин}

Данный файл содержит перечень предметов и всех компетенций, которые им соответствуют. Также для каждого предмета его код из актуальной версии рабочих учебных планов. Данный файл содержит перечень предметов и преподавателей (лекторов и семинаристов на текущий год). Данный перечень может меняться от года к году. На его основе (с использованием описанных ниже файлов рабочих программ) могут быть, в частности, автоматически сформированы планы работы преподавателей.
\begin{verbatim}
{
  "Disciplines" : [
  {
  "Id" : "Б1.О.01",
  "Name" : "Современные проблемы прикладной математики и информатики",
  "Competence" : "УК- 1; ОПК - 2; ПК - 2",
  "Lecturer" : "Семенов И. В. ",
  "Seminarian" : "Семенов И. В. "
  },
  {
  "Id" : "Б1.О.02",
  "Name" : "Иностранный язык",
  "Competence" : "УК - 4; УК - 5",
  "Lecturer" : "Тимофеев А. В. ",
  "Seminarian" : "Тимофеев А. В. "
  },
  {
  "Id" : "Б1.О.03",
  "Name" : "Современные компьютерные технологии",
  "Competence" : "УК- 1; ОПК - 4; УКЦ-1; УКЦ-2",
  "Lecturer" : "Лазарев В. В. ",
  "Seminarian" : "Лазарев В. В. "
  },
  {
  "Id" : "Б1.О.04",
  "Name" : "Метод конечных элементов",
  "Competence" : "УК - 3; УК - 6; ПК - 5; ПК - 6; ПК - 7",
  "Lecturer" : "Сидоров М. А. ",
  "Seminarian" : "Сидоров М. А. "
  },
  {
  "Id" : "Б1.О.05",
  "Name" : "Основы математической теории переноса",
  "Competence" : "ОПК - 1; ОПК - 2; ПК - 5",
  "Lecturer" : "Гичук А. В. ",
  "Seminarian" : "Гичук А. В. "
  },
  {
  "Id" : "Б1.О.06",
  "Name" : "Обобщенные функции",
  "Competence" : "ОПК - 1; ОПК - 3; ПК - 5",
  "Lecturer" : "Лутиков И. В. ",
  "Seminarian" : "Лутиков И. В. "
  },
  {
  "Id" : "Б1.О.07",
  "Name" : "Численные методы газовой динамики",
  "Competence" : "ОПК - 1; ОПК - 3; ПК - 1; ПК - 6; ПК - 7; ПК - 8",
  "Lecturer" : "Наумов А. О.",
  "Seminarian" : "Наумов А. О."
  },
  {
  "Id" : "Б1.О.08",
  "Name" : "Методы аппроксимации задач математической физики на нерегулярных
            сетках",
  "Competence" : "ОПК - 1; ПК - 5",
  "Lecturer" : "Сидоров М. А. ",
  "Seminarian" : "Сидоров М. А. "
  },
  {
  "Id" : "Б1.О.09",
  "Name" : "Методы интегральных преобразований",
  "Competence" : "ОПК - 1; ОПК - 3; ПК - 5; ПК - 10",
  "Lecturer" : "Тихонов А. В. ",
  "Seminarian" : "Тихонов А. В. "
  },
  {
  "Id" : "Б1.В.01",
  "Name" : "Разностные схемы решения многомерных уравнений механики сплошной
            среды в эйлеровых переменных",
  "Competence" : "ПК - 9; ПК - 10; ПК-7.1; ПК-7.2",
  "Lecturer" : "Янилкин Ю.В. ",
  "Seminarian" : "Янилкин Ю.В. "
  },
  {
  "Id" : "Б1.В.02",
  "Name" : "Математическое моделирование задач механики сплошных сред на
            высокопроизводительных ЭВМ",
  "Competence" : "ПК - 9; ПК - 10; ПК-7.1; ПК-7.2",
  "Lecturer" : "Городничев А. В. ",
  "Seminarian" : "Городничев А. В. "
  },
  {
  "Id" : "Б1.В.ДВ.01.01",
  "Name" : "Методы распараллеливания задач математической физики на
            многопроцессорных ЭВМ",
  "Competence" : "ПК - 2; ПК - 3; ПК - 10",
  "Lecturer" : "Воропинов А. А. ",
  "Seminarian" : "Воропинов А. А. "
  },
  {
  "Id" : "Б1.В.ДВ.01.02",
  "Name" : "Высокопроизводительные вычисления",
  "Competence" : "ПК - 2; ПК - 3; ПК - 10",
  "Lecturer" : "Воропинов А. А. ",
  "Seminarian" : "Воропинов А. А. "
  },
  {
  "Id" : "Б1.В.ДВ.02.01",
  "Name" : "Численные методы теории переноса",
  "Competence" : "ПК - 1; ПК - 6; ПК - 7",
  "Lecturer" : "Гичук А. В. ",
  "Seminarian" : "Гичук А. В. "
  },
  {
  "Id" : "Б1.В.ДВ.02.02",
  "Name" : "Решение уравнения переноса методом Монте-Карло",
  "Competence" : "ПК - 1; ПК - 6; ПК - 7",
  "Lecturer" : "Гичук А. В. ",
  "Seminarian" : "Гичук А. В. "
  }
  ]
}
\end{verbatim}

\subsection{Файлы рабочих программ}

Предлагаемая в данном разделе структура файлов рабочих программ дисциплин — основная часть предлагаемого подхода. По результатам анализа ряда рабочих программ предлагается следующий перечень содержательных разделов файлов рабочих программ. Примеры значений полей приводятся на примере дисциплины «Численные методы газовой динамики», входящей в образовательную программу магистратуры по специальности 01.04.02 «Прикладная математика и информатика»
\subsubsection{Факультет}
Ключевое слово — Faculty, значение — строка.\\
\verb/  "Faculty" : "Физико-технический факультет"/

\subsubsection{Кафедра}
Ключевое слово — Chair, значение — строка.\\
\verb/  "Chair" : "Кафедра прикладной математики"/

\subsubsection{Наименование дисциплины}
Ключевое слово — Name, значение — строка.\\
\verb/  "Name" : "Численные методы газовой динамики"/

\subsubsection{Направление подготовки (специальности)}
Ключевое слово — TrainingDirection, значение — строка.\\
\verb/  "TrainingDirection" : "01.04.02 <<Прикладная математика и информатика>>"/

\subsubsection{Наименование образовательной программы}
Ключевое слово — EducationProgram, значение — строка.
\begin{verbatim}
  "EducationProgram" : "<<Высокопроизводительные вычисления и технологии
                        параллельного программирования>>"
\end{verbatim}

\subsubsection{Квалификация (степень) выпускника}
Ключевое слово — Degree, значение — строка. \\
\verb/  "Degree" : "магистр"/

\subsubsection{Данные из актуального учебного плана}
Ключевое слово — Semester, значение — массив, элементами которого являются структуры, показывающие, в каких семестрах, какое количество часов, сколько лекций-практик-лабораторных-самостоятельных, форма промежуточного контроля (зачёт, экзамен, зачёт с оценкой). Соответствующие ключевые слова:
\begin{verbatim}
[
  {"Id", "ZET", "VolumeHours", "LectureHours", "PracticeHours",
   "LaboratoryHours", "PrivateWork", "ControlForm" },...
]
\end{verbatim}
Если дисциплина читается в одном семестре — массив состоит из одного элемента.
\begin{verbatim}
"Semester" : [
  {"Id" : 2, "ZET" : 2,  "VolumeHours" : 72, "LecturesHours" : 0,
   "PracticeHours" : 32, "LaboratoryHours" : 0, "PrivateWork" : 40,
   "ControlForm" : "Зачёт"},
  {"Id" : 3, "ZET" : 3, "VolumeHours" : 108, "LecturesHours" : 0,
   "PracticeHours" : 32, "LaboratoryHours" : 0, "PrivateWork" : 40,
   "ControlForm" : "Экзамен"}
]
\end{verbatim}

\subsubsection{Аннотация}
Ключевое слово — Annote, значение — строка.\\
\verb/  "Annote" : "..."/

\subsubsection{Цели и задачи освоения учебной дисциплины}
Ключевое слово — Aims, значение — строка.
\begin{verbatim}
    "Aims" : "Целями освоения учебной дисциплины <<Численные методы газовой
динамики>> является подготовка студентов к их профессиональной деятельности.
Общеизвестно, что задачи механики сплошной среды (и особо - газодинамики)
занимают центральное место в профессиональной работе выпускников кафедры
прикладной математики.\n\nЗадачи дисциплины - дать основы:\\begin{itemize}
\n\\itemзаконов сохранения - массы, количества движения, энергии;\n\\item
особенностей газодинамических течений: характеристики, инвариантность Римана,
существование разрывных течений.\n\\item построения и исследования разностных
схем для уравнений переноса и ГД.\\end{itemize}"
\end{verbatim}

\subsubsection{Место учебной дисциплины в структуре ООП ВО}
ООП ВО -- основная образовательная программа высшего образования.

Ключевое слово — PlaceInStructrure, значение — строка.
\begin{verbatim}
    "PlaceInStructure" : "Дисциплина «Численные методы газовой динамики»
(Б1.В.ОД.3) является обязательной дисциплиной вариативной части образовательной
программы магистратуры по специальности 01.04.02 «Прикладная математика и
информатика»"
\end{verbatim}

\subsubsection{Предметы, необходимые для успешного освоения дисциплины}
Ключевое слово — Background, значение — строка.
\begin{verbatim}
    "Background" : "Освоение дисциплины предполагает у студентов владение
рядом базовых дисциплин, в первую очередь это «Математический анализ»,
«Механика сплошной среды», «Уравнения математической физики», «Численные методы»,
«Теория функций комплексных переменных».\n\nДля успешного освоения дисциплины
необходимы знания по курсам общей физики и университетскому курсу математики.
Необходимо иметь навыки в программировании и пользовании прикладными программами,
обладать знаниями основ теории переноса, а также ударно-волновых процессов в
материальных средах"
\end{verbatim}

\subsubsection{Структура и содержание учебной дисциплины}
Ключевое слово — StructureAndContents, значение — массив, элементами которого являются структуры, показывающие, какие темы, какое количество часов лекций-практик, разделы данной темы (массив, при необходимости), текущий контроль: форма, неделя, контролируемые компетенции и знания — умения — навыки (рекомендуется оставлять пустые поля, в этом случае имеется в виду контроль всех компетенций для данной дисциплины). Соответствующие ключевые слова:
\begin{verbatim}
"StructureAndContents" : [
{ "ThemeTitle", "LectureHours", "PracticeHours", "Sections" : [ ],
  "Control":{ "Form", "Week", "Competence", "ZUN" } }, ...
]
\end{verbatim}
\begin{verbatim}
  "StructureAndContents" : [
    { "ThemeTitle" : "Элементы газовой динамики",
      "LectureHours" : "",
      "PracticeHours" : "",
      "Sections" : [
        "Подходы Лагранжа и Эйлера к изучению движения сплошной среды",
        "Интегральная форма законов сохранения для фиксированной и произвольной
         массы газа",
        "Дифференциальная форма законов сохранения, различные представления
         уравнения энергии",
        "Линеаризованный аналог уравнений газовой динамики. Уравнения акустики",
        "Характеристики уравнений газовой динамики, инварианты Римана",
        "Разрывные решения, соотношения на разрывах",
        "Структура фронта ударной волны при введении счетной вязкости",
        "Задача о выдвижении поршня из газа. Центрированная волна разрежения",
        "Задача Римана о распаде произвольного разрыва"
      ],
      "Control" : {
        "Form" : "",
        "Week" : "",
        "Competence" : "",
        "ZUN" : ""
      }
    },
    { "ThemeTitle" : "Основные понятия теории разностных схем",
      "LectureHours" : "",
      "PracticeHours" : "",
      "Sections" : [
        "Разностные сетки. Сеточные функции",
        "Аппроксимация, сходимость, устойчивость",
        "Методы исследования устойчивости разностных схем",
        "Первое дифференциально приближение",
        "Уравнение переноса как простейший пример квазилинейного гиперболического
         уравнения",
        "Дисперсия разностных схем"
      ],
      "Control" : {
        "Form" : "",
        "Week" : "",
        "Competence" : "",
        "ZUN" : ""
      }
    },
    { "ThemeTitle" : "Принципы построения разностных схем для уравнений
                      газодинамики",
      "LectureHours" : "",
      "PracticeHours" : "",
      "Sections" : [
        "Некоторые разностные схемы для расчёта одномерных газодинамических
         течений",
        "Полностью консервативная разностная схема для расчёта одномерных
         течений газа",
        "О полной консервативности схемы <<крест>>"
      ],
      "Control" : {
        "Form" : "",
        "Week" : "",
        "Competence" : "",
        "ZUN" : ""
      }
    },
    { "ThemeTitle" : "Исследование устойчивости разностных схем, аппроксимирующих
                      уравнения газодинамики в одномерном плоском случае",
      "LectureHours" : "",
      "PracticeHours" : "",
      "Sections" : [
        "Уравнения акустики – модель уравнений газодинамики",
        "Исследование устойчивости некоторых разностных схем для уравнений
         акустики",
        "Исследование устойчивости некоторых разностных схем для уравнений
         газовой динамики"
      ],
      "Control" : {
        "Form" : "",
        "Week" : "",
        "Competence" : "",
        "ZUN" : ""
      }
    },
    { "ThemeTitle" : "Реализация разностных схем для расчёта газодинамических
                      течений",
      "LectureHours" : "",
      "PracticeHours" : "",
      "Sections" : [
        "Общие замечания о методах решения"
      ],
      "Control" : {
        "Form" : "",
        "Week" : "",
        "Competence" : "",
        "ZUN" : ""
      }
    },
    { "ThemeTitle" : "Pешение многомерных задач газовой динамики",
      "LectureHours" : "",
      "PracticeHours" : "",
      "Sections" : [
        "Пространственные лагранжевые координаты",
        "Дискретизация задачи. Построение начальной регулярной счетной сетки",
        "Поддержание, корректировка счетной сетки",
        "Пересчет величин при корректировке счетной сетки"
      ],
      "Control" : {
        "Form" : "",
        "Week" : "",
        "Competence" : "",
        "ZUN" : ""
      }
    },
    { "ThemeTitle" : "Методика расчета двумерных задач газовой динамики в
                      переменных Лагранжа (методика Д)",
      "LectureHours" : "",
      "PracticeHours" : "",
      "Sections" : [
        "Аппроксимация основных уравнений"
      ],
      "Control" : {
        "Form" : "",
        "Week" : "",
        "Competence" : "",
        "ZUN" : ""
      }
    },
    { "ThemeTitle" : "Построение полностью консервативной разностной схемы для
                      расчета двумерных осесимметричных газодинамических
                      течений в переменных Лагранжа",
      "LectureHours" : "",
      "PracticeHours" : "",
      "Sections" : [
        "Аппроксимация основных уравнений"
      ],
      "Control" : {
        "Form" : "",
        "Week" : "",
        "Competence" : "",
        "ZUN" : ""
      }
    },
    { "ThemeTitle" : "Метод концентраций расчета нестационарных течений
                      многокомпонентной сплошной среды",
      "LectureHours" : "",
      "PracticeHours" : "",
      "Sections" : [
        "Замечания о моделях среды",
        "Типы замыкающих соотношений"
      ],
      "Control" : {
        "Form" : "",
        "Week" : "",
        "Competence" : "",
        "ZUN" : ""
      }
    }
  ]
\end{verbatim}


\subsubsection{Образовательные технологии, используемые в реализации дисциплины}
Ключевое слово — Technologies, значение — строка.\\
\verb/  "Technologies" : "не предусмотрены"/

\subsubsection{Основная литература}
Ключевое слово — BasicLiterature, значение — массив строк.
\begin{verbatim}
"BasicLiterature" : [
    "Самарский А. А., Попов Ю. П. Разностные методы решения задач газовой
     динамики. М.: Наука, 1992 г.",
    "Ландау Л. Д., Лифшиц Е. М. Гидродинамика. М.: Наука, 1986 г.",
    "Курант Р., Фридрихс К. Сверхзвуковые течения и ударные волны. М.: ИИЛ, 1960 г.",
    "Станюкович К. П. Неустановившиеся движения сплошной среды. М.: Наука, 1971 г.",
    "Овсянников Л. В. Лекции по основам газовой динамики. М.: Наука, 1984 г.",
    "Зельдович Я. Б., Райзер Ю. П. Физика ударных волн и высокотемпературных
     гидродинамических явлений. М.: Наука, 1966 г.",
    "Рождественский Б. Л., Яненко Н. Н. Системы квазилинейных уравнений и их
     приложения к газовой динамике, М., Наука, 1968 г."
]
\end{verbatim}

\subsubsection{Дополнительная литература}
Ключевое слово — AdditionalLiterature, значение — массив строк.
\begin{verbatim}
"AdditionalLiterature" : [
    "Поттер Д. Вычислительные методы в физике. М.: Мир, 1975 г.",
    "Шокин Ю. П., Яненко Н.Н. Метод дифференциального приближения. Применение
     к газовой динамике. Новосибирск: Наука (с.о.), 1985 г.",
    "Годунов С. К., Рябенький В. С. Разностные схемы. М.: Наука, 1973 г.",
    "Численное решение многомерных задач газовой динамики. Ред. Годунов С. К.,
     М.: Наука, 1976",
    "Вычислительные методы в гидродинамике, ред. Б. Олдер, С. Фернбах,
     М. Ротенберг, М.: Мир, 1967",
    "Вопросы математического моделирования, вычислительной математики и
     информатики, ред. С. М. Бахрах, В. Н. Михайлов, И. Д. Софронов,
     Арзамас-16: Изд-во ВНИИЭФ, 1994",
]
\end{verbatim}


\subsubsection{Материально – техническое обеспечение учебной дисциплины}
Ключевое слово — MaterialSupport, значение — строка.\\
\verb/    "MaterialSupport" : «отсутствует»/

\subsubsection{Методические рекомендации студентам по организации изучения дисциплины}
Ключевое слово — MetodicalRecommendations, значение — строка.
\begin{verbatim}
    "MetodicalRecommendations" : "\begin{itemize}\n\item в первом разделе
уделить особое внимание различным видам записи законов сохранения и структуре
решения газодинамической системы уравнений;\n\item во втором разделе рассмотреть
основные принципы исследования свойст разностных схем;\n\item в третьем разделе
особое внимание уделить консервативности разностных схем;\n\item в четвертом
разделе рассмотреть методику исследования устойчивости разностных схем для
газодинамической системы уравнений;\n\item в пятом разделе показать особенности
реализации разностных схем для уравнений газодинамики;\n\item в шестом --
остановиться на основных принципах выбора счетной сетки и построения, разностных
схем для многомерных задач газовой динамики;\n\item в седьмом рассмотреть
разностные уравнении лежащие в основе старейшей лагранжевой методики Д;\n\item в
восьмом обратить внимание на достоинства и недостатки явных и неявных разностных
схем;\n\item в девятом рассмотреть основные принципы метода частиц;\n\item
в десятом уделить особое внимание методу концентраций.»
\end{verbatim}

\subsubsection{Паспорт фонда оценочных средств}
Ключевое слово — FOSPassport, значение — строка.\\
\verb/  "FOSPassport" : ""/

\subsubsection{Область применения}
Ключевое слово — FOSApplicationDomain, значение — строка.\\
\verb/  "FOSApplicationDomain" : ""/

\subsubsection{Цели и задачи фонда оценочных средств}
Ключевое слово — FOSAims, значение — строка.\\
\verb/  "FOSAims" : ""/

\subsubsection{Домашние задания}
Ключевое слово — Homework, значение — массив структур: [{ToDo, Weeks}].\\
\verb/  "Homework" : [ { "ToDo", "Weeks"}, ... ]/

\subsubsection{Перечень вопросов для устного опроса}
Ключевое слово — OralQuiz, значение — массив строк.\\
\verb/  "OralQuiz" : ""/

\subsubsection{Перечень задач}
Ключевое слово — Exersizes, значение — массив строк.
\begin{verbatim}
  "Exersizes" : [
    "Определить степень сжатия вещества сильной ударной волной, если начальное
     давление вещества близко к нулю, а показатель адиабаты равен 1.4",
    "Определить степень сжатия вещества сильной ударной волной, если начальное
     давление вещества близко к нулю, а показатель адиабаты равен 5/3",
    "Провести исследование устойчивости схемы «назад по потоку»
     $\\frac{u_k^{n+1}-u_k^n}{\\tau}+c\\frac{u_k^n-u_{k-1}^n}{h}=0$ для уравнения
     переноса $\\frac{\\partial u}{\\partial t}+c\\frac{\\partial u}{\\partial t}=0$:
     a) методом гармоник, б) используя оценку нормы разностного оператора",
    "Провести исследование устойчивости схемы $\\frac{u_k^{n+1}-u_k^n}{\\tau}+
     c\\frac{u_k^{n+1}-u_{k-1}^{n+1}}{h}=0$ для уравнения переноса
     $\\frac{\\partial u}{\\partial t}+c\\frac{\\partial u}{\\partial t}=0$:
     a) методом гармоник, б) используя оценку нормы разностного оператора",
    "Для разностной схемы «назад по потоку» $\\frac{u_k^{n+1}-u_k^n}{\\tau}+
     c\\frac{u_k^n-u_{k-1}^n}{h}=0$ для уравнения переноса
     $\\frac{\\partial u}{\\partial t}+c\\frac{\\partial u}{\\partial t}=0$
     выполнить: a) определить порядок аппроксимации, б) выписать первое
     дифференциальное приближение, в) провести исследование устойчивости
     методом гармоник",
    "Для разностной схемы $\\frac{u_k^{n+1}-u_k^n}{\\tau}+c
     \\frac{u_k^{n+1}-u_{k-1}^{n+1}}{h}=0$ для уравнения переноса
     $\\frac{\\partial u}{\\partial t}+c\\frac{\\partial u}{\\partial t}=0$
     выполнить: a) определить порядок аппроксимации, б) выписать первое
     дифференциальное приближение, в) провести исследование устойчивости
     методом гармоник",
    "Провести исследование дисперсионных свойств схемы «назад по потоку»
     $\\frac{u_k^{n+1}-u_k^n}{\\tau}+c\\frac{u_k^n-u_{k-1}^n}{h}=0$ для
     уравнения переноса $\\frac{\\partial u}{\\partial t}+
     c\\frac{\\partial u}{\\partial t}=0$",
    "Провести исследование дисперсионных свойств схемы
     $\\frac{u_k^{n+1}-u_k^n}{\\tau}+c\\frac{u_k^{n+1}-u_{k-1}^{n+1}}{h}=0$
     для уравнения переноса $\\frac{\\partial u}{\\partial t}+
     c\\frac{\\partial u}{\\partial t}=0$",
    "Провести исследование дисперсионных свойств схемы
     $\\frac{u_k^{n+1}-u_k^n}{\\tau}+c\\frac{u_{k+1}^{n+1}-u_{k-1}^{n+1}}{2h}=0$
     для уравнения переноса $\\frac{\\partial u}{\\partial t}+
     c\\frac{\\partial u}{\\partial t}=0$",
    "Провести исследование устойчивости системы разностных уравнений:
     $$\\frac{p_k^{n+1}-p_k^n}{\\tau}+a\\frac{u_k^{n}-u_{k-1}^{n}}{h}=0,\\
     \\frac{u_k^{n+1}-u_k^n}{\\tau}+a\\frac{p_{k+1}^{n}-p_{k}^{n}}{h}=0.$$",
    "Получить необходимое и достаточное условие устойчивости для схемы
     Лакса-Вендроффа: $$u_k^{n+1}=u_k^n-\\frac{a\\tau}{2h}(u^n_{k+1}-u^n_{k-1})+
     \\frac{a^2\\tau^2}{2h^2}(u_{k+1}^n-2u_k^n+u_{k-1}^n).$$",
    "Провести исследование устойчивости системы разностных уравнений:
     $$\\frac{p_k^{n+1}-p_k^n}{\\tau}+a\\frac{u_k^{n}-u_{k-1}^{n}}{h}=0,\\
     \\frac{u_k^{n+1}-u_k^n}{\\tau}+a\\frac{p_{k+1}^{n+1}-p_{k}^{n+1}}{h}=0.$$",
    "Провести исследование устойчивости системы разностных уравнений:
     $$\\frac{p_k^{n+1}-p_k^n}{\\tau}+a\\frac{u_k^{n+1}-u_{k-1}^{n+1}}{h}=0,\\
     \\frac{u_k^{n+1}-u_k^n}{\\tau}+a\\frac{p_{k+1}^{n+1}-p_{k}^{n+1}}{h}=0.$$",
    "Провести исследование устойчивости явной и неявной схем для уравнения
     переноса $\\frac{\\partial u}{\\partial t}+c\\frac{\\partial u}{\\partial t}
     =0$, имеющие шаблоны: $\\perp$, $\\top$",
    "Исследовать законы сохранения для ГД схемы\n
     $$\n\\rho_n^k\\frac{u_k^{n+1}-u_k^n}{\\tau}=
     -\\frac{p_{k+1/2}^{n+1/2}-p_{k-1/2}^{n+1/2}}{(\\Delta x)^n},\\
     \n\\frac{\\rho_{k+1/2}^{n+1}-\\rho_{k+1/2}^n}{\\tau}=-\\rho_{k+1/2}^{n+1}
     \\frac{\\left(u_{k+1}^{n+1/2}-u_{k}^{n+1/2}\\right)}{(\\Delta x)^n}\n$$\n
     $$\n\\frac{e_{k+1/2}^{n+1}-e_{k+1/2}^n}{\\tau}=
     -\\frac{p_{k+1/2}^{n+1/2}}{\\rho_{k+1/2}^n}\\frac{u_{k+1}^{n+1/2}-u_{k}^{n+1/2}}
     {(\\Delta x)^n},\\ p_{k+1/2}^n=(\\gamma-1)\\rho_{k+1/2}^ne_{k+1/2}^n\n$$",
    "Получить необходимое условие устойчивости для разностной схемы\n
    $$\n\\frac{u_k^{n+1}-u_k^n}{\\tau}=-\\frac{p_{k+1/2}^{n+1}-p_{k-1/2}^{n+1}}
    {\\Delta s},\\ \n\\frac{v_{k+1/2}^{n+1}-v_{k+1/2}^n}{\\tau}=
    \\frac{\\left(u_{k+1}^{n+1/2}-u_{k}^{n+1/2}\\right)}{\\Delta s}\n$$\n
    $$\ne_{k+1/2}^{n+1}-e_{k+1/2}^n=-p_{k+1/2}^{n+1}(v_{k+1/2}^{n+1}-v_{k+1/2}^n),\\
    p_{k+1/2}^{n+1}=(\\gamma-1)e_{k+1/2}^{n+1}/v_{k+1/2}^{n+1}\n$$"
  ]
\end{verbatim}

\subsubsection{Перечень тестов}
Ключевое слово — Tests, значение — массив структур [{Theme, Variant, Questions[]}].\\
\verb/  "Tests" : []/

\subsubsection{Перечень контрольных работ}
Ключевое слово — ControlWorks, значение — массив структур [{Theme, Variant, Questions[]}].\\
\verb/  "ControlWorks" : []/

\subsubsection{Перечень вопросов к зачету}
Ключевое слово — ZachetQuestions, значение — массив структур [{Semester, Questions[]}]\\
\verb/  "ZachetQuestions" : []/

\subsubsection{Перечень вопросов к экзамену}
Ключевое слово — ExamQuestions, значение — массив структур [{Semester, Questions[]}]
\begin{verbatim}
  "ExamQuestions" : [ {
    "Semester" : 3,
    "Questions" : [
      "Подходы Лагранжа и Эйлера к изучению движения сплошной среды. Связи между
       этими подходами",
      "Интегральная форма законов сохранения для фиксированной массы газа",
      "Интегральная форма законов сохранения для произвольного объёма газа",
      "Дифференциальная форма законов сохранения в переменных Эйлера",
      "Дифференциальная форма законов сохранения в переменных Лагранжа",
      "Уравнение энергии, различные формы представления",
      "Уравнение акустики",
      "Одномерные течения газа",
      "Различные виды уравнений одномерной газодинамики. Интегральная форма
       одномерных уравнений",
      "Массовые лагранжевы координаты, уравнения газодинамики в лагранжевых
       массовых координатах",
      "Характеристики уравнений газовой динамики",
      "Инварианты Римана",
      "Разрывные решения, соотношения на разрывах",
      "Соотношения на ударной волне в идеальном газе, адиабата Гюгонио",
      "Искусственная вязкость. Структура фронта ударной волны при наличии
       искусственной вязкости",
      "Задача о выдвижении поршня. Центрированные волны разрежения",
      "Задача Римана о распаде произвольного разрыва",
      "Принципы построения разностных схем",
      "Аппроксимация, устойчивость, сходимость",
      "Первое дифференциальное приближение разностной схемы. Г-форма и П-форма.
       Свойства первого дифференциального приближения",
      "Разностные схемы для уравнения переноса, исследование устойчивости",
      "Монотонность разностных схем, примеры",
      "Дисперсионные свойства разностных схем",
      "Методы исследования устойчивости разностных схем. Анализ нормы разностного
       оператора",
      "Метод гармоник исследования устойчивости",
      "Исследование разностных схем для одномерного уравнения переноса",
      "Принципы построения разностных схем для уравнений газодинамики",
      "Разностные схемы для расчёта одномерных газодинамических течений",
      "Схема типа <<крест>> для расчёта одномерных газодинамических течений",
      "Полностью консервативная разностная схема для расчёта одномерных
       газодинамических течений",
      "Уравнения акустики – модель уравнений газодинамики. Разностная схема,
       исследование устойчивости",
      "Исследование устойчивости разностных схем одномерной газодинамики",
      "Исследование устойчивости полностью консервативной схемы расчёта
       одномерных течений газа",
      "Исследование устойчивости разностной схемы типа <крест>>",
      "Исследование устойчивости разностной схемы типа “крест” с линейной
       вязкостью",
      "Неявные разностные схемы расчёта одномерных газодинамических течений.
       Исследование устойчивости",
      "Ограничение временного шага при сквозном расчёте ударных волн",
      "Реализация разностных схем расчёта одномерных газодинамических течений",
      "Дискретизация задачи (двумерная геометрия)",
      "Двумерные лагранжевы координаты",
      "Построение начальной регулярной счётной сетки",
      "Поддержание и корректировка счётной сетки",
      "Методика расчёта двумерных задач газовой динамики в переменных Лагранжа.
       (Методики Д.)",
      "Характерный размер счётного многоугольника",
      "Сглаживание возмущений поля скоростей",
      "Явная полностью консервативная разностная схема расчёта двумерных
       газодинамических течений",
      "Неявная полностью консервативная разностная схема расчёта двумерных
       течений газа",
      "Расчёт газодинамических течений в лагранжево–эйлеровой постановке
       (случай однородной среды)",
      "Метод концентраций",
      "Замыкающие соотношения в методе концентраций"
    ]
  } ],
\end{verbatim}

\subsubsection{Перечень билетов к экзамену}
Ключевое слово — ExamTickets, значение — массив структур [{Semester, Tickets[]}]
\begin{verbatim}
  "ExamTickets" : [ {
    "Semester" : 3,
    "Tickets" : [
        "Подходы Лагранжа и Эйлера к изучению движения сплошной среды.
         Связи между этими подходами. Интегральные формы законов сохранения",
        "Дифференциальная форма законов сохранения в переменных Эйлера и
         Лагранжа. Различные виды представления уравнения энергии",
        "Одномерные течения газа. Различные виды уравнений одномерной
         газодинамики. Интегральная форма одномерных уравнений",
        "Массовые лагранжевы координаты, уравнения газодинамики в лагранжевых
         массовых координатах",
        "Характеристики уравнений газовой динамики. Инварианты Римана",
        "Разрывные решения, соотношения на разрывах. Адиабата Гюгонио",
        "Искусственная вязкость. Структура фронта ударной волны при наличии
         искусственной вязкости",
        "Задача о выдвижении поршня. Центрированные волны разрежения",
        "Задача Римана о распаде произвольного разрыва",
        "Первое дифференциальное приближение разностной схемы. Г-форма и П-форма.
         Свойства первого дифференциального приближения",
        "Дисперсионные свойства разностных схем",
        "Методы исследования устойчивости разностных схем",
        "Полностью консервативная разностная схема для расчёта одномерных
         газодинамических течений",
        "Схема типа “крест”. Аппроксимация, устойчивость, выполнение законов
         сохранения",
        "Разностная схема методики расчёта двумерных задач газовой динамики в
         переменных Лагранжа. (Методики Д)",
        "Консервативная разностная схема методики расчёта двумерных
         осесимметричных течений (Методики ЛЭГАК)",
        "Метод концентраций. Замыкающие соотношения в методе концентраций"
      ]
    } ]
\end{verbatim}

\subsubsection{Шкалы оценки образовательных учреждений}
Ключевое слово — GradeScale, значение — строка.
\begin{verbatim}
  "GradeScale" : "Рейтинговая оценка знаний является интегральным показателем
качества теоретических и практических знаний и навыков студентов по дисциплине
и складывается из оценок, полученных в ходе текущего контроля и промежуточной
аттестации.\n\nРезультаты текущего контроля и промежуточной аттестации подводятся
по шкале балльно-рейтинговой системы.\n\nШкала каждого контрольного мероприятия
лежит в пределах от 0 до установленного максимального балла включительно.
Итоговая аттестация по дисциплине оценивается по 100-балльной шкале и
представляет собой сумму баллов, заработанных студентом при выполнении заданий
в рамках текущего и промежуточного контроля.\n\nИтоговая оценка выставляется
в соответствии со следующей шкалой:"
\end{verbatim}

\section{Примеры автоматически решаемых задач}
После того, как подготовлены файлы с описанием компетенций ({\tt competence.json}), с перечнем дисциплин -- преподавателей -- покрываемых компетенций ({\tt lecturers.json}) и папка, содержащая набор файлов с описаниями рабочих программ дисциплин ({\tt DisciplinesFolder}), появляется возможность автоматического  решения целого ряда разнообразных задач, связанных с сопровождением программы обучения. Для решения нам будет удобно пользоваться языком Python.

\subsection{Получение списка дисциплин с отсутствующими рабочими программами}
В качестве первого примера приведём скрипт для вывода дисциплин из {\tt lecturer.json}, для которых в данной папке отсутствует рабочая программа:\\
\begin{verbatim}
#--------FindAbsentDisciplines.py----------
#!/usr/bin/env python
import json
import os

def FindAbsentDiscipline( lecturerFile : str, disciplinesFolder : str ) -> None:
# READING THE DISCIPLINES LIST WITH WORKPROGRAM
  rpnames = os.listdir( "./" + disciplinesFolder )
  RP = []
  for r in rpnames:
  rpname = "./" + disciplinesFolder + "/" + r
  if r.count( ".json" ) > 0:
  INP2 = open( rpname, "r", encoding = "utf-8" )
  rr = json.load( INP2 )
  RP.append( rr["Name"] )
#CHECK IF DISCIPLINES ARE IN THE LIST
  INP1 = open( "./" + lecturerFile, "r", encoding = "utf-8" )
  dis = json.load( INP1 )
  for i in range( len( dis["Disciplines"] ) ):
  dd = dis["Disciplines"][i]["Name"]
  if dd not in RP:
  print( dd )

if __name__ == '__main__':
  FindAbsentDiscipline( "lecturer.json", "DisciplinesFolder" )
\end{verbatim}

В данном скрипте вывод дисциплин производится на экран, однако легко их выводить и в текстовый файл.

\subsection{Аннотация образовательной программы}

Для создания аннотации образовательной программы необходимо:

1) сформировать титульный лист

2) для каждой дисциплины из перечня {\tt lecturers.json} включить (отдельным разделом) данные из учебного плана, аннотацию, цели и задачи изучения дисциплины, место учебной дисциплины в структуре ООП ВО, формируемые компетенции и планируемые результаты обучения.

Титульный лист формируется на основе названий направления подготовки, образовательной программы и квалификации (степени) выпускника
с помощью строковой функции типа
\begin{verbatim}
def TitlePageString( disfilename : str ) -> str:
  INP = open( disfilename, "r", encoding = "utf-8" )
  dis = json.load( INP )
  return """
...
""" %(dis["TrainingDirection"], dis["EducationProgram"], dis["Degree"])
\end{verbatim}
Здесь в качестве {\tt disfilename} подаётся имя файла рабочей программы какой-либо дисциплины из программы (они все содержат необходимую информацию).

Информация по каждой дисциплине формируется с помощью функций:\\
формирования таблицы нагрузки для дисциплины в необходимом формате (на основе считанной в формате JSON дисциплины {\tt dis})\\
\verb'def TableHours( dis ) -> str: ...'\\
перечня номеров универсальных компетенций из общего списка {\tt UC}, которые формируются дисциплиной {\tt dis}\\
\verb'def UniversalCompetenceNum( dis, UC ):...'\\
перечня номеров профессиональных компетенций из общего списка {\tt PC}, которые формируются дисциплиной {\tt dis}\\
\verb'def ProfessionalCompetenceNum( dis, PC ):...'\\
формирования таблицы универсальных компетенций, которые формируются дисциплиной {\tt dis}, в необходимом формате\\
\verb'def TableUniversalCompetence( dis, UC ) -> str:...'\\
формирования таблицы профессиональных компетенций, которые формируются дисциплиной {\tt dis}, в необходимом формате\\
\verb'def TableProfessionalCompetence( dis, PC ) -> str: ...'\\
формирования раздела в аннотации для дисциплины {\tt dis} в необходимом формате\\
\verb'def StringForDisc( disfile : str, competencefile : str ) -> str:...'\\

Аннотация собирается из титульного листа и разделов для тех дисциплин из {\tt lecturers.json}, которые есть в папке с рабочими программами.

\begin{verbatim}
def WriteProgram( disciplinesFolder : str, competenceFile : str, lecturerFile : str ) -> str:
  result = ""
  rpnames = os.listdir( "./" + disciplinesFolder )
  INP1 = open( "./" + lecturerFile, "r", encoding = "utf-8" )
  alldis = json.load( INP1 )
  for i in range( len( alldis["Disciplines"] ) ):
  dd = alldis["Disciplines"][i]["Name"]
  for r in rpnames:
  rpname = "./" + disciplinesFolder + "/" + r
  if r.count( ".json" ) > 0:
   INP2 = open( rpname, "r", encoding = "utf-8" )
   rr = json.load( INP2 )
   if dd == rr["Name"]:
  if result == "":
  result += TitlePageString( rpname )
  result += StringForDisc( dname, compfilename )
  result += "\n\\end{document}"

if __name__ == '__main__':
  OUT = open( "annotation.tex" ,"w", encoding = "utf-8")
  OUT.write( WriteProgram( "DisciplinesFolder", "competence.json", "lecturer.json" ) )
\end{verbatim}
Пример аннотации образовательной программы, собранной указанным способом, можно посмотреть в \cite{Annote010402}.

\subsection{Определение аудиторной нагрузки преподавателей}
В файле {\tt lecturers.json} содержится информация о преподаваемых в текущем учебном году дисциплинах и их преподавателях, а в файлах рабочих программ -- информация о нагрузке, можно реализовать скрипт для расчёта учебной нагрузки преподавателей по данной образовательной программе.

При необходимости результат можно выдавать в виде таблиц, используя для их автоматического построения формат CSV (\cite{CSV}).

\subsection{Создание рабочих программ дисциплин}
Для оформления рабочих программ дисциплин по текущим правилам необходимо, аналогично переборам в предыдущих пунктах, реализовать для рабочей программы и фонда оценочных средств каждой дисциплины из образовательной программы формирование соответствующих разделов. После того, как соответствующие программы будут реализованы, последующие изменения и актуализации потребуют минимальных трудозатрат.


\newpage
\section*{Заключение}
\addcontentsline{toc}{section}{Заключение}

Для обеспечения высокого уровня подготовки образовательным учреждениям необходимо постоянно контролировать и оптимизировать механизм управления учебно-ор\-га\-ни\-за\-ци\-он\-ной деятельностью, что влечет за собой автоматизацию процессов в организации.

Актуальность выбранной темы выпускной квалификационной работы подтверждается ее практической значимостью.

В ходе выполнения исследовательской работы был разработан альтернативный подход к подготовке и сопровождению массивов документов для сопровождения программ высшего профессионального образования, основанный на:

- разработке формата описания рабочих программ дисциплин в виде текстовых файлов, размеченных с использованием формата JSON;

- скриптах обработки полученных файлов для решения типовых задач на языке Python;

- формировании необходимых документов в виде \LaTeX-файлов.

В ходе работы были проработаны возможности сложного автоматического форматирования, подготовки таблиц, использования спецсимволов, формул (что необходимо, например, для описания рабочих программ по математическим дисциплинам). Вместе с тем, целый ряд задач в рамках указанного подхода может быть решён за счёт формирования документов в виде текстовых файлов с простым форматированием, либо CSV-файлов, которые легко обработать в редакторах таблиц.

На основе материалов настоящего исследования представляется возможность улучшения взаимодействия между участниками учебного процесса вуза. Также для снижения возможных рисков необходимо предусмотреть своевременное наполнение базы входными данными, загрузку актуальной нормативно-справочной информации с электронных информационных ресурсов Министерства образования и науки РФ.

\newpage
\begin{thebibliography}{99}
\addcontentsline{toc}{section}{Список использованных источников}
\bibitem{FGOS} Федеральный государственный образовательный стандарт высшего образования - магистратуры по направлению подготовки 01.04.02 Прикладная математика и информатика //
Федеральные государственные образовательные стандарты : официальный сайт. -- 2018. --
URL: \href{https://fgos.ru/fgos/fgos-01-04-02-prikladnaya-matematika-i-informatika-13/?ysclid=lx7d7kyvu3278701072}{https://fgos.ru/fgos/fgos-01-04-02-prikladnaya-matematika-i-informatika-13/?ysclid=lx7d7kyvu3278701072} (дата обращения 29.04.2024).
\bibitem{OS} Образовательный стандарт высшего образования национального исследовательского ядерного университета <<МИФИ>>, уровень высшего образования: магистратура по направлению подготовки
01.04.02 Прикладная математика и информатика // СарФТИ НИЯУ МИФИ : официальный сайт. -- 2024. -- URL: \href{https://sarfti.ru/sveden/files/000057.pdf}{https://sarfti.ru/sveden/files/000057.pdf} (дата обращения 29.04.2024).
\bibitem{JSON} JSON // Электронная энциклопедия : [сайт]. -- 2024. -- URL: \href{https://ru.m.wikipedia.org/wiki/JSON}{https://ru.m.wikipedia.org/wiki/JSON}
(дата обращения 03.05.2024).
\bibitem{UTF-8} UTF-8 // Электронная энциклопедия : [сайт]. -- 2024. -- URL: \href{https://ru.m.wikipedia.org/wiki/UTF-8}{https://ru.m.wikipedia.org/wiki/UTF-8} (дата обращения 14.05.2024).
\bibitem{Python} Python // Электронная энциклопедия : [сайт]. -- 2024. -- URL: \href{https://ru.m.wikipedia.org/wiki/Python}{https://ru.m.wikipedia.org/wiki/Python}  (дата обращения 03.05.2024).
\bibitem{CSV} CSV // Электронная энциклопедия : [сайт]. -- 2024. -- URL: \href{https://ru.m.wikipedia.org/wiki/CSV}{https://ru.m.wikipedia.org/wiki/CSV} (дата обращения 01.06.2024).
\bibitem{LaTeX} Львовский С. М. Набор и верстка в пакете \LaTeX{}. М.: МЦНМО, 2014. 400 с.
\bibitem{KM} Характеристика образовательной программы: компетентностная модель выпускника магистратуры по направлению подготовки 01.04.02 Прикладная математика и информатика
// СарФТИ НИЯУ МИФИ : официальный сайт. -- 2024. -- URL: \href{https://sarfti.ru/sveden/files/000269.pdf}{https://sarfti.ru/sveden/files/000269.pdf} (дата обращения 22.05.2024).
\bibitem{Annote010402} Аннотации к образовательным программам // СарФТИ НИЯУ МИФИ : официальный сайт. -- 2024. -- URL: \href{https://sarfti.ru/sveden/files/001832.pdf}{https://sarfti.ru/sveden/files/001832.pdf} (дата обращения 28.05.2024).

\end{thebibliography}

\end{document}

1. Балдин, К.В Информационные системы в экономике: Учебник / К.В Балдин, В.Б. Уткин. - М.: Дашков и К, 2015. - 395 c.
2. Блиновская, Я.Ю. Введение в геоинформационные системы: Учебное пособие / Я.Ю. Блиновская, Д.С. Задоя. - М.: Форум, НИЦ ИНФРА-М, 2013. - 112 c.
3. Бодров, О.А Предметно-ориентированные экономические информационные системы / О.А Бодров. - М.: ГЛТ, 2013. - 244 c.
4. Буреш, О.В. Интеллектуальные информационные системы управления социально-экономическими объектами / О.В. Буреш, М.А. Жук. - М.: Красанд, 2016. - 192 c.
5. Варфоломеева, А.О. Информационные системы предприятия: Учебное пособие / А.О. Варфоломеева, А.В. Коряковский, В.П. Романов. - М.: НИЦ ИНФРА-М, 2013. - 283 c.
6. Веллинг Л., Томсон Л. Разработка веб-приложений с помощью PHP и MySQL. – М.: Вильямс, 2015. – 848с.
7. Вдовин, В.М. Предметно-ориентированные экономические информационные системы: Учебное пособие / В.М. Вдовин, Л.Е. Суркова и др. - М.: Дашков и К, 2016. - 388 c.
8. Гвоздева, В.А. Информатика, автоматизированные информационные технологии и системы: Учебник / В.А. Гвоздева. - М.: ИД ФОРУМ, НИЦ ИНФРА-М, 2013. - 544 c.
9. Горбенко, А.О. Информационные системы в экономике: Учебное пособие / А.О. Горбенко. - М.: БИНОМ. Лаборатория знаний, 2014. - 292 c.
10. Грекул В.И., Денищенко Г.Н. Проектирование информационных систем. – М.: Интернет-университет информационных технологий, 2015. - 304 с.
11. Емельянов, С.В. ИТ и вычислительные системы: Вычислительные системы. Математическое моделирование. Прикладные аспекты информатики / С.В. Емельянов. - М.: Ленанд, 2015. - 96 c.
12. Кириллов В.В., Громов Г.Ю. Введение в реляционные базы данных. – СПб.: БХВ-Петербург, 2016. – 464с..
13. Косиненко, Н.С. Информационные системы и технологии в экономике: Учебное пособие / Н.С. Косиненко, И.Г. Фризен. - М.: Дашков и К, 2015. - 304 c.
14. Кувшинов, М.С. Информационные системы в экономике. Управление эффективностью банковского бизнеса / М.С. Кувшинов. - М.: КноРус, 2013. - 176 c.

15. Миков, А.И. Информационные процессы и нормативные системы в IT: Математические модели. Проблемы проектирования. Новые подходы / А.И. Миков. - М.: КД Либроком, 2013. - 256 c.
16. Одинцов, Б.Е. Информационные системы управления эффективностью бизнеса: Учебник и практикум для бакалавриата и магистратуры / Б.Е. Одинцов. - Люберцы: Юрайт, 2016. - 206 c.
17. Плаксин, М. Тестирование и отладка программ - для профессионалов будущих и настоящих. - М.:Бином, 2017. - 168 с.
18. Рыжко, А.Л. Информационные системы управления производственной компанией: Учебник для академического бакалавриата / А.Л. Рыжко, А.И. Рыбников, Н.А. Рыжко. - Люберцы: Юрайт, 2016. - 354 c.
19. Сырецкий, Г.А. Уткин, В.Б. Информационные системы в экономике: Учебник для студентов высших учебных заведений / В.Б. Уткин, К.В. Балдин. - М.: ИЦ Академия, 2014. - 288 c.
20. Федотова, Е.Л. Информационные технологии и системы: Учебное пособие / Е.Л. Федотова. - М.: ИД ФОРУМ, НИЦ ИНФРА-М, 2013. - 352 c.
21. Ясенев, В.Н. Информационные системы и технологии в экономике: Учебное пособие / В.Н. Ясенев. - М.: ЮНИТИ, 2014. - 560 c.
22. ГОСТ 34.601-90.1. Информационная технология. Комплекс стандартов на автоматизированные системы. Автоматизированные системы стадии создания (утвержден постановлением Госстандарта СССР от 29.12.1990 № 3469) [Электронный ресурс] // СПС КонсультантПлюс.
23. ГОСТ 34.602-89. Межгосударственный стандарт. Информационная технология. Комплекс стандартов на автоматизированные системы. Техническое задание на создание автоматизированной системы (утвержден постановлением Госстандарта СССР от 24.03.1989 № 661) [Электронный ресурс] // СПС КонсультантПлюс.
24. ГОСТ Р 50-34.126-92. Рекомендации. Информационная технология Правила проведения работ при создании автоматизированных систем, (утверждены и введены в действие постановлением Госстандарта СССР от 03.02.1992 № 99) [Электронный ресурс] // СПС КонсультантПлюс.
25. Государственная программа «Информационное общество (2011- 2020 годы)» [Электронный ресурс]: [утв. Постанов. Правительства РФ от 15.04.2014 г. N 313 (в ред. ПП РФ от 21.02.2015 N 157, от 17.06.2015 N 602] // СПС КонсультантПлюс.
26. Государственная программа Российской Федерации «Развитие образования» на 2013-2020 годы» [Электронный ресурс]: [утв. постановлением Правительства РФ от 15.04.2014 г. N 295] // СПС КонсультантПлюс.
27. Материалы [Электронный ресурс]: ФГОС ВО // Портал Федеральных государственных образовательных стандартов высшего образования [Офиц. Сайт]. URL: http://fgosvo.ru/fgosvo/92/91/4 (дата обращения: 05.05.2018).
28. Научная библиотека диссертаций и авторефератов disserCat http://www.dissercat.com/content/sistemnyi-analiz-i-optimizatsiya-tekhnologicheskogo-protsessa-avtomatizatsii-sostavleniya-ra\#ixzz5GyNZeOAD
29. Нормативные документы [Электронный ресурс]: Организационная структура НИУ «БелГУ // НИУ «БелГУ» [Офиц. Сайт]. https://www.bsu.edu.ru/bsu/structure/section.php?IBLOCK_ID=78\&SECTION_ID=5278 (дата обращения: 28.05.2018).
30. Нормативные документы [Электронный ресурс]: Устав федерального государственного автономного образовательного учреждения высшего образования «Белгородский государственный национальный исследовательский университет» (новая редакция) Приказ Минобрнауки России от 30.12.2015 №1547 «О ФГАОУ ВО НИУ «БелГУ»: Устав: Приказ Минобрнауки России от 23.08.2017 №830 «О внесении изменений в Устав ФГАОУ ВО НИУ «БелГУ» https://www.bsu.edu.ru/bsu/resource/officialdocs/sections.php?ID=174
31. Об образовании в Российской Федерации [Электронный ресурс]: Федер. закон [принят Гос. Думой 29.12.2020, ред. от 02.03.2016] // СПС КонсультантПлюс.
32. Обзор изменений федерального закона от 29.12.2012 N 273-ФЗ «Об образовании в Российской Федерации» [Электронный ресурс] // СПС КонсультантПлюс.
33. Об информации, информационных технологиях и о защите информации [Электронный ресурс]: Ф.З. [принят Гос. Думой 27.07.2006] // СПС КонсультантПлюс.
34. Перечень образовательных программ, реализуемых в НИУ «БелГУ» [Электронный ресурс] // [Офиц. Сайт]. URL: http://dekanat.bsu.edu.ru/blocks/bsu_nabor/nabor.php?facid=10200 (дата обращения: 05.05.2018).
35. Положение об учебно-методическом отделе (утв. 28.09.2015 г.) Электронный ресурс]: https://www.bsu.edu.ru/bsu/resource/officialdocs/sections.php?ID=158
36. Положение о кафедре НИУ "БелГУ" (утв. 28.09.2015 г.) [Электронный ресурс]: https://www.bsu.edu.ru/bsu/resource/officialdocs/sections.php?ID=158
37. Сведения об образовательной организации [Электронный ресурс]: Уровни образования. Формы и сроки обучения. Образовательные программы. Срок действия государственной аккредитации. Учебные планы. Аннотации РП // Белгородский государственный национальный исследовательский


